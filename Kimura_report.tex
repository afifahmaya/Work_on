\documentclass[]{article}

\usepackage{amsmath}
\usepackage{amssymb}
\usepackage{amsthm}
\usepackage{amsfonts}
\usepackage{color}
\usepackage{graphicx}

\newtheorem{remark}{Remark}
\newtheorem{prop}{Proposition}
\newtheorem{theo}{Theorem}
\newtheorem{defi}{Definition}
\newtheorem{ex}{Example}
\newtheorem{note}{Notes}

\newcommand{\R}{\mathbb{R}}
\renewcommand{\thesection}{\arabic{section}.}

% Title Page
\title{Finite Difference Method (FDM) and \\Finite Element Method (FEM)}
\author{Afifah Maya Iknaningrum \\ 1715011053}


\begin{document}
\maketitle

\section{Introduction}
Partial Differential Equation (PDE) is an equation with two or more variables and its partial derivative. For integer $ k \geq 1 $ and $ U $ is open subset of $ \R^n $, Evans, L.C.(2010) define PDE as an expression of the form
\[ F(D^k u(x), D^{k-1} u(x), \dots , Du(x), u(x), x)=0, \]
that is called a \textit{k\textsuperscript{th}-order} partial differential equation, where
\[ F : \R^{n^{k}} \times \R^{n^{k-1}} \times \dots \times \R^{n} \times \R \times U \rightarrow \R \]
is given and
\[ u : U \rightarrow \R \]
is the unknown. Some example of PDE is Heat Equations, Wave Equations, Elasticity Equations, Maxwell Equations, Navier-Stokes Equations, etc.

In the study of these equations, there are some PDEs that the analytical solution is not easy to get. To solve it, usually the solution is approached by numerical method. Such as Boundary (Integral) Element Method, Particle Method, Spectral Method, Finite Difference Method (FDM) and Finite Element Method (FEM), etc. In this report, we will discuss about the comparison of FDM and FEM.

To understand the FDM and FEM, we would like to show how these method can be used to solve Poisson Equation for bounded domain $ \Omega \subset \R^2 $,
\begin{equation} \label{Poisson}
\begin{cases}
- \bigtriangleup u = f(x) &, x\in\Omega\\
u = g(x) &, x \in \partial \Omega
\end{cases}
\end{equation}
with $ x=(x_{1},x_{2}) $, $ u=u(x)=u(x_{1},x_{2}) $, and Laplacian $ \bigtriangleup = \dfrac{\partial^2}{\partial x_{1}^2} + \dfrac{\partial^2}{\partial x_{2}^2} $.

\section{Finite Difference Method}
We consider mesh as shown below with mesh size $ \Delta x  = \Delta y = h>0 $ and $ (x,y) = (x_{1},x_{2}) $.\\
\begin{figure}[h!]
	\centering
	\includegraphics[width=0.5\linewidth]{../Downloads/FiniteDifference_twod_mesh}
	\caption{}
	\label{fig:finitedifferencetwodmesh}
\end{figure}
\\The solution of PDE $ u(\xi_{ij}) $ is approximated by solution of FDM $ u_{ij} $ for each point $ \xi_{ij} = (ih,jh) \in \R^2 $. Using one of FDM, \underline{central difference scheme}, the solution of Poisson equation in (\ref{Poisson}) can be obtained. 

Let $ \Omega = (0,1) \times (0,1) $ and $ h=\dfrac{1}{N} $ where $ N $ is number of devider of domain. We define $ w_{h} := \{ \xi_{ij} | \ \xi_{ij} \in \Omega \} $ and $ \gamma_{h} := \{ \xi_{ij} | \ \xi_{ij} \in \partial \Omega \} $. Then (\ref{Poisson}) can be approximated by
\begin{equation}\label{poisson_discrete}
\begin{cases}
- \dfrac{u_{i+1,j}+u_{i-1,j}+u_{i,j+1}+u_{i,j-1}-4 \ u_{ij}}{h^2} = f_{ij} &, (\xi_{ij}\in w_{h}) \\
u_{ij} = g_{ij} &,(\xi_{ij}\in\gamma_{h})
\end{cases}
\end{equation}
where we assume $ f \in C(\bar{\Omega}) $ and $ g \in C(\partial \Omega) $,
$ \begin{cases}
f_{ij} := f(\xi_{ij}) \\ g_{ij} := g(\xi_{ij})
\end{cases} $. Then we could get the approximated value of $ u $ by solving equation (\ref{poisson_discrete}).

\section{Finite Element Method}
Consider Poisson equation in (\ref{Poisson}), by variational principle, there exist a unique solution $ u = argmin \ E(v) $, for $ v \in H_{0}^{1}(\Omega) $, $ v=0 $ at boundary, where
\[ E(v) := \dfrac{1}{2} \int_{\Omega} |\nabla v|^2 \ dx - \int_{\Omega} f \ v \ dx \]
with $ \nabla v = \Big[ \dfrac{\partial v}{\partial x_{1}} \ , \ \dfrac{\partial v}{\partial x_{2}} \Big] $. There exist weak solution for equation (\ref{Poisson}) if and only if exist $ u = argmin \ E(v) $. 

Let $ \Omega $ be a polygon. We define $ \bar{\Omega} = \bigcup\limits_{K \in T_{h}} K $ with $ K $ is closed triangle and $ T_{h} $ is triangular division. First, we define the vector space
\[ V_{h} :=  \{ u \in C(\Omega) \ | \ u_{|K} \text{: a linear function, } \forall K \in T_{h}, u_{| \partial \Omega}=0 \}. \]
which is $ V_{h} \subset H_{0}^{1}(\Omega) $ with $ dim \ V_{h} = N $. Then, we use notation $ \{ P_{i} \}_{i=1}^{M} $ as the nodal point of $ T_{h} $. For $ \{ P_{i} \}_{i=1}^{N} \subset \{ P_{i} \}_{i=1}^{M} \ : \ P_{i} \in \Omega $ is interior nodal point.

Here we use $ V_{h} $ as approximation for space $ H_{0}^{1}(\Omega) $. Then to approximate PDE weak solution $ u = argmin \ E(v) $ for $ v \in H_{0}^{1}(\Omega) $, we approach its value by $ u_{h} = argmin \ E(v_{h}) $ for $ v_{h} \in V_{h} $. Using FEM, we can obtain a linear system
\begin{equation}\label{linear_sys}
\mathbf{A} \mathbf{u} = \mathbf{b}
\end{equation}
where $ \mathbf{A} \in \R^{N \times N} $ is symmetric and sparse matrix, and $ \mathbf{u}, \mathbf{b} \in \R^{N} $.

There are many method we can use to solve the linear system. One of the method that usually used is Conjugate Gradient (CG) Method, because it works well to solve (\ref{linear_sys}). Then by sloving the linear system (\ref{linear_sys}), we obtain the approximated value of $ u $.

As summary, the procedure of FEM we need to process first, is mesh generation. After that, assemble $ \mathbf{A} $ and $ \mathbf{b} $ from the mesh $ T_{h} $. Then, solve the linear system using a linear solver to obtain the solution.


\section{Comparison}
Here are some difference between FDM and FEM :\\ \\
FDM
\begin{enumerate}
	\item FDM is approximation of the differential operator by finite difference.
	\item The function is approximated in grid points.
	\item Difficult to apply for not rectangular domain.
\end{enumerate}
FEM
\begin{enumerate}
	\item The domain is approximated by triangular mesh.
	\item Approximate the function space under variational structure.
	\item Easy to apply for curved domain
\end{enumerate}

As we can see, the FEM in two dimesional  or three dimensional problem is much more powerful than FDM.

\end{document}          
