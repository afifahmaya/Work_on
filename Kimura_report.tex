\documentclass[]{report}

\usepackage{amsmath}
\usepackage{amssymb}
\usepackage{amsthm}
\usepackage{amsfonts}
\usepackage{color}
\usepackage{graphicx}

\newtheorem{remark}{Remark}
\newtheorem{prop}{Proposition}
\newtheorem{theo}{Theorem}
\newtheorem{defi}{Definition}
\newtheorem{ex}{Example}
\newtheorem{note}{Notes}

\newcommand{\R}{\mathbb{R}}

% Title Page
\title{Finite Difference Method (FDM) and \\Finite Element Method (FEM)}
\author{Afifah Maya Iknaningrum}


\begin{document}
\maketitle

\section{Introduction}
Partial Differential Equation is \dots . For example is Heat Equations, Wave Equations, Elasticity Equations, Maxwell Equations, Navier-Stokes Equations, etc.

Because there are some PDEs that the analytical solution is not easy to get, we usually approach the solution by numerical method. For example is Boundary (Integral) Element Method, Particle Method, Spectral Method, Finite Difference Method (FDM) and Finite Element Method (FEM), etc. In this report, we will discuss about the comparison of FDM and FEM.

\section{Finite Difference Method}

\section{Finite Element Method}

\section{Comparison}
%buat table

FDM
\begin{enumerate}
	\item FDM is approximation of the differential operator by finite difference.
	\item The function is approximated in grid points.
	\item Difficult to apply for not rectangular domain.
\end{enumerate}

FEM
\begin{enumerate}
	\item The domain is approximated by triangular mesh.
	\item Approximate the function space under variational structure.
	\item Easy to aply for curved domain
\end{enumerate}

As we can see, the FEM in two dimesional  or three dimensional problem is much more powerful than FDM.

\end{document}          
