\documentclass{beamer}

\usepackage{default}
\usepackage{hanging}
\usepackage{amsmath}
\usepackage{amsthm}
\usepackage{amsfonts}
\usepackage{enumerate}

\usetheme{Warsaw}

\usepackage{etoolbox}
\usepackage{xpatch}
\makeatletter
\patchcmd\beamer@@tmpl@frametitle{\insertframetitle}{%
	\ifnumgreater{\arabic{section}}{0}{%
		\thesection.%
		\ifnumgreater{\arabic{subsection}}{0}{% 
			\thesubsection.%
			\ifnumgreater{\arabic{subsubsection}}{0}{%
				\thesubsubsection.%
			}{}%
		}{}%
	}{}%
	~\insertframetitle{}{}%
}%
\makeatother

\setbeamercovered{highly dynamic}
\newcounter{saveenumi}
\newcommand{\seti}{\setcounter{saveenumi}{\value{enumi}}}
\newcommand{\conti}{\setcounter{enumi}{\value{saveenumi}}}
\resetcounteronoverlays{saveenumi}

\usepackage{tikz}
\usetikzlibrary{shapes.geometric, arrows}
\tikzstyle{startstop} = [rectangle, draw, rounded corners, minimum width=3cm, minimum height=1cm, text centered, draw=black]
\tikzstyle{process} = [rectangle, draw, minimum width=3cm, minimum height=1cm, text centered, text width=4cm, draw=black]
\tikzstyle{line} = [draw, -latex']

\newcommand{\R}{\mathbb{R}}

\begin{document}
\title{Progress Report}
\author{Afifah Maya Iknaningrum}
\date{2018}
\institute{Kanazawa University}

\begin{frame}
\titlepage
\end{frame}

\begin{frame}{Incompressible Navier-Stokes}
\begin{block}{Strong Form}
	We want to find \[(u,p) : \Omega \times (0,T) \rightarrow \R^3 \times \R\] where $ u $ is unknown velocity and $ p $ is unknown pressure such that
	\begin{equation}\label{Navier-Stokes}
	\begin{cases}
	\dfrac{\partial u}{\partial t} + (u \cdot \nabla) u - \nu \bigtriangleup u + \nabla p = f & \text{ in } \Omega \times (0,T)\\
	\nabla \cdot u = 0 & \text{ in } \Omega \times (0,T)\\
	u = 0 & \text{ on } \partial \Omega \times (0,T)\\
	u = u^0 & \text{ in } \Omega \text{, at } t=0
	\end{cases}
	\end{equation}
	where $ f : \Omega \times (0,T) \rightarrow \R^3 $ and $ u^0 : \Omega \rightarrow \R^3 $ are given functions, choosing $ \nu > 0 , \nu = 1$ is a viscosity.
\end{block}
\end{frame}

\begin{frame}{Incompressible Navier-Stokes}
\begin{block}{Weak Form}
	We want to find $ \{ (u,p)(t) \in V \times Q ; t \in (0.T) \} $ such that for $ t \in (0,T) $
	\begin{equation} \label{NS_Weak} \nonumber
	\begin{cases}
	\big( \dfrac{\partial u}{\partial t} + (u \cdot \nabla)u,v \big) + a(u,v) + b(v,p) + b(u,q) = (f,v) & , \\ \hspace{5cm} \forall(v,q)\in V\times Q \\ u=u^{0} , \hspace{4cm} t=0
	\end{cases}
	\end{equation}
	\begin{eqnarray}\nonumber
	a(u,v) &=& \nu \int_{\Omega} \nabla u : \nabla v \ dx \\ \nonumber
	b(v,q) &=& - \int_{\Omega} (\nabla \cdot v) q \ dx \\ \nonumber
	V &=& H_{0}^{1}(\Omega, \R^d) = H_{0}^{1}(\Omega)^d \\ \nonumber
	Q &=& \{ q\in L^2(\Omega) ; \int_{\Omega} q \ dx=0 \}.
	\end{eqnarray}
\end{block}
\end{frame}

\begin{frame}{3D Discretization}
\begin{block}{First order in time}
	Before applying to FreeFEM++, we need to discritize $ \dfrac{\partial u_{i}}{\partial t} + (u \cdot \nabla)u_{i} $ part, where $ dt $ as time increment.
	\[ \dfrac{\partial u_{i}}{\partial t} + (u \cdot \nabla)u_{i} \approx \dfrac{u_{i}^{n}-u_{i}^{n-1}(X_{1}(u^{n-1},dt))}{dt} + O(dt+h) \]
\end{block}
\begin{block}{Second order in time / Adam-Bashforth Method}
	\begin{eqnarray}\nonumber
		&\dfrac{\partial u_{i}}{\partial t} + (u \cdot \nabla)u_{i} \approx\\ \nonumber &\dfrac{3u_{i}^{n}-4u_{i}^{n-1}(X_{1}(\tilde{u}^{n-1},dt))+u_{i}^{n-2}(X_{1}(\tilde{u}^{n-1},2dt))}{2\ dt} + O(dt^2+h^2)
	\end{eqnarray}
\end{block}
\end{frame}

\begin{frame}
\begin{block}{where}
	\[ X_{1}(u^{n-1},dt)(x) = x - u^{n-1}(x)\ dt\]\\
	\[\tilde{u}^{n-1}_{i} = 2u_{i}^{n-1}-u_{i}^{n-2}\]
\end{block}
\begin{block}{with stabilization term}
	With $ \delta>0 $ and $ h $ as mesh size
	\[ C_{i}(p,q) = \delta \sum_{k} h_{k}^{2}(\nabla p, \nabla q)_{k} \]
\end{block}
\end{frame}

\begin{frame}{Error estimate}
\begin{block}{$ L^{2} $}
	\[\| u_{h}^{n}-u^{n} \|_{L^\infty(L^2)} =  max \ \| u_{h}^{n}-u^{n} \|_{L^{2}}\]
\end{block}
\begin{block}{$ H_{1} $}
	\[\| u_{h}^{n}-u^{n} \|_{L^\infty(H^{1})} = max \sqrt{\| u_{h_{1}}^{n}-u_{1}^{n} \|_{L^2(\Omega)}^{2} + \| \nabla (u_{h_{1}}^{n}-u_{1}^{n}) \|_{L^2(\Omega)}^{2}}\]
\end{block}
\end{frame}

\begin{frame}{Cylindrical domain simulation}
\begin{block}{Exact solution}
		\begin{eqnarray}\nonumber
		u &=& (u_{1},u_{2},u_{3}) \\ \nonumber
		u_{1} &=& -\cos(x_{1}) \sin(x_{2}) \cos(x_3) e^{-2t}\\ \nonumber
		u_{2} &=& -\sin(x_{1}) \cos(x_{2}) \cos(x_3) e^{-2t}\\ \nonumber
		u_{3} &=& 0 \\ \nonumber
		p&=& \dfrac{1}{4} e^{-4t} (\cos(2x_1)+\cos(2x_2)+\cos(2x_3))
		\end{eqnarray}
		such that equation (\ref{Navier-Stokes}) is satisfied with $ f = (f_{1},f_{2},f_3) $. With $ f_{1} = -\cos(x_1) \sin(x_2) \cos(x_3) e^{-2t} $, $ f_{2} = -\sin(x_1) \cos(x_2) \cos(x_3) e^{-2t}  $, and $ f_{3} = -(\dfrac{1}{4})e^{-4t}\sin(2x_3)(2\cos(2x_3)+1)   $
\end{block}
\end{frame}

\begin{frame}
\begin{figure}
	\centering
	\includegraphics[width=1\linewidth]{NS_3D/error_cyl}
	\caption{}
	\label{fig:errorcyl}
\end{figure}
\end{frame}

\begin{frame}
\begin{figure}
	\centering
	\includegraphics[width=1\linewidth]{NS_3D/magnitude_cyl}
	\caption{}
	\label{fig:magnitudecyl}
\end{figure}
\end{frame}

\begin{frame}{Tornado simulation on cylindrical domain}
	\begin{block}{Domain and initial condition}
		Taking $ a=1/8, \epsilon_{i} =1, \beta_{i}=1 \ (i=1,\dots,6) $, with domain $ \Omega = \{ x=(x,y,z) \in \R^3 ; -a\leq z\leq 4a, \ \sqrt{x^2+y^2}<1 \} $ and $ u=0 $ on boundary.
		\begin{eqnarray}\nonumber
		\psi(a,\epsilon,\sigma) &=& (a^{2}+\epsilon)^{\sigma}\\ \nonumber
		u_{z} &=& \psi(r,\epsilon_{1},-\beta_{1})\psi(z,\epsilon_{2},-\beta_{2})\\ \nonumber
		\rho &=& \psi(r,\epsilon_{3},-\beta_{3})\psi(z,\epsilon_{4},\beta_{4})\\ \nonumber
		u_{0} &=& \psi(r,\epsilon_{5},-\beta_{5})\psi(z,\epsilon_{6},-\beta_{6}) \text{\hspace{1cm} (with swirl)}\\ \nonumber
		u_{0} &=& 0 \text{ \hspace{5cm}(no swirl) }\\ \nonumber
		u_{r} &=& sign(z)\rho u_{z}
		\end{eqnarray}
	\end{block}
\end{frame}

\begin{frame}
\begin{figure}
	\centering
	\includegraphics[width=1\linewidth]{NS_3D/magnitude_tornado}
	\caption{Max $ v $ of tornado simulation every time step}
	\label{fig:magnitudetornado}
\end{figure}
\end{frame}

\begin{frame}{Tornado simulation on curved cylindrical domain}
Using Gmsh
	\begin{figure}
		\centering
		\includegraphics[width=1\linewidth]{NS_3D/1curved}
		\caption{}
		\label{fig:1curved}
	\end{figure}
\end{frame}

\begin{frame}{Tornado simulation on curved cylindrical domain}
Using FreeFEM++ (applying Kazunori's ideas)
\begin{figure}
	\centering
	\includegraphics[width=1\linewidth]{NS_3D/curved}
	\caption{}
	\label{fig:curved}
\end{figure}

\end{frame}

\end{document}
