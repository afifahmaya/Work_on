\documentclass[a4paper,10pt]{article}
\usepackage[a4paper, hmargin={2cm,2cm}, vmargin={1.5cm,1.5cm}]{geometry}
\usepackage{amsmath}
\usepackage{amsthm}
\usepackage{amsfonts}
\usepackage{color}
\usepackage{graphicx}

\newcommand{\R}{\mathbb{R}}

\begin{document}

\section{18-04-18}
This research is collaboration with Prof. Yoneda from University of Tokyo.\\
Main equation discussed is \textbf{Navier-Stokes equation} that usually discusses in fluid, for example air.

\subsection{Navier-Stokes Equation}

\subsubsection{General Problem}
For dimension $ d=2,3,... $ (usually $ 2 $ or $ 3 $) and $ T>0 $, we want to find \[(u,p) : \Omega \times (0,T) \rightarrow \R^d \times \R\] where $ u $ is unknown velocity and $ p $ is unknown pressure such that
\begin{equation}\label{Navier-Stokes}
\begin{cases}
\dfrac{\partial u}{\partial t} + (u \cdot \nabla) u - \nu \bigtriangleup u + \nabla p = f & \text{ in } \Omega \times (0,T)\\
\nabla \cdot u = 0 & \text{ in } \Omega \times (0,T)\\
u = 0 & \text{ on } \partial \Omega \times (0,T)\\
u = u^0 & \text{ in } \Omega \text{, at } t=0
\end{cases}
\end{equation}
where $ f : \Omega \times (0,T) \rightarrow \R^d $ and $ u^0 : \Omega \rightarrow \R^d $ are given functions, $ \nu > 0 $ is a viscosity.

From equation (\ref{Navier-Stokes}) we can see that $ \dfrac{\partial u}{\partial t} + (u \cdot \nabla) u $ is the \textbf{convection part} that explain the movement of fluid. This part, contain \textbf{nonlinear term} $ (u \cdot \nabla) u $. We can also see, that $ \nu \bigtriangleup u $ (similar to Heat equation) is the \textbf{diffusion part}. In the second equations, $ \nabla \cdot u = 0 $ explained the \textbf{incompressible condition} of fluid. \\ \\
\textbf{Incompressible condition :}
\[ \nabla \cdot u = div \ u = 0 \Leftrightarrow \text{ fluid is incompressible} \]
means that the total amount of body does not change. By
\[ 0 = \int_{V} \nabla \cdot u \ dx = \int_{\partial V} u \cdot n \ ds \]
means that the energy that comes in and comes out is same and the normal component of velocity is $ 0 = \int_{\partial V} u \cdot n \ ds $ where $ n $ is the normal vector works on boundary. \vspace{3cm} \\
\textbf{Convection effect :}\\
\textbf{[simple explanation]} Let $ \phi^{0}(x), c>0 $ is given. Consider $ \phi(x,t) = \phi^{0}(x-ct) $, that represent the movement of function without changing the shape. \vspace{4cm} \\
at $ t=0 $ we have $ \phi(x,0) = \phi^{0}(x) $ ; at at $ t=1 $ we have $ \phi(x,1) = \phi^{0}(x-c) $ ; at $ t=2 $ we have $ \phi(x,2) = \phi^{0}(x-2c) $ as shown above.

If we differentiate $ \phi $ over $ x $ and $ t $, then we obtain 
\[ \begin{cases}
\dfrac{\partial \phi}{\partial t} (x,t) = \phi^{0}\prime (x-ct) \ (-c) = -c \ \phi^{0}\prime (x-ct) \text{; the initial function}\\
\dfrac{\partial \phi}{\partial x} (x,t) = \phi^{0}\prime (x-ct) \text{; moves to right with velocity c}
\end{cases}. \]
From the above relation, we get 
\[ \dfrac{\partial \phi}{\partial t} + c \ \dfrac{\partial \phi}{\partial x} = 0. \]
If we consider velocity, $ c \leftarrow u $ then $ \dfrac{\partial \phi}{\partial t} + u \ \dfrac{\partial \phi}{\partial x} = 0 $. Next, for \textbf{multidimensional convection equation}, $ \dfrac{\partial}{\partial x} \leftarrow \nabla $, $ \dfrac{\partial \phi}{\partial t} + (u \cdot \nabla) \phi= 0 $. In Navier-Stokes equations, if we $ \phi \leftarrow u_{i} $, then the first two term in first equation of equation (\ref{Navier-Stokes}). \\ \\
\textbf{[explanation]} Let $ u : \Omega \times (0,T) \rightarrow \R^d $ is given.
\[ \dfrac{\partial \phi}{\partial t} (x,t) + [(u \cdot \nabla) \phi](x,t) = 0 , \ (x,t) \in \Omega \times (0,T). \]
Let us consider the position of a fluid particle that satisfy
\[ \begin{cases}
X\prime (t) &= u (X(t),t), \ \forall t \\
X(t\star) &= x
\end{cases}. \]
Calculate
\begin{eqnarray}\nonumber
\dfrac{d}{dt}[\phi(X(t),t)] &=& (\nabla \phi)(X(t),t) \cdot X \prime (t) + \dfrac{\partial \phi}{\partial t}(X(t),t) \\ \nonumber
&=& [(u \cdot \nabla)\phi](X(t),t) + \dfrac{\partial \phi}{\partial t}(X(t),t) \\ \nonumber
&=& \Big[ \dfrac{\partial \phi}{\partial t}+(u \cdot \nabla)\phi \Big] (X(t),t).
\end{eqnarray}
If we set $ t=t\star $, then
\[ \dfrac{d}{dt}[\phi(X(t),t)]_{|t=t\star} = \Big[ \dfrac{\partial \phi}{\partial t}+(u \cdot \nabla)\phi \Big] (x,t\star) = 0 \]
or means that the function value does not change if it is changes by the velocity $ u $, or called \textbf{characterictic line trajectory of particle}.

\subsubsection{3D Problem}
For $ d=3 $, then $ u = \left[ {\begin{array}{c} u_{1} \\ u_{2} \\ u_{3}
\end{array} } \right] $ such that for $ (i=1,2,3) $ we have
\[ \dfrac{\partial u_{i}}{\partial t} + (u \cdot \nabla) u_{i} - \nu \bigtriangleup u_{i} + [\nabla p]_{i} = f_{i} \]
where
\begin{eqnarray}\nonumber
(u \cdot \nabla) u_{i} &=& \Big( \left[ 
\begin{array}{c}
u_{1} \\ u_{2} \\ u_{3}
\end{array} 
\right] \cdot \left[
\begin{array}{c}
\partial_{1} \\ \partial_{2} \\ \partial_{3}
\end{array}
\right] \Big) u_{i} \\ \nonumber
&=& (u_{1} \partial_{1} + u_{2} \partial_{2} + u_{3} \partial_{3}) u_{i} \\ \nonumber
&=& u_{1}\dfrac{\partial u_{i}}{x_{1}} + u_{2}\dfrac{\partial u_{i}}{x_{2}} + u_{3}\dfrac{\partial u_{i}}{x_{3}}
\end{eqnarray}
and
\begin{eqnarray} \nonumber
\bigtriangleup u_{i} &=& \dfrac{\partial^{2}u_{i}}{\partial x_{1}^2} + \dfrac{\partial^{2}u_{i}}{\partial x_{2}^2} + \dfrac{\partial^{2}u_{i}}{\partial x_{3}^2}.
\end{eqnarray}
\textbf{Note :}we have $ u_{1}, u_{2}, u_{3}, p $ as four unknown functions and four equations (as first equation defined for three $ u $ and second equation), then we could find the solution.

\subsection{Research Topic}
We will study about axisymmetric flow (example : air). Consider sylindrical domain for first. We do two simulation, first : with the initial velocity with velocity concentration is in the center of axis, second : we include swirl, like tornado type velocity.

In this research it is proved that \textit{if there is blow up, then there is swirl}. But has not proved that there is some blow-up phenomena ($ \exists(x\star, t\star), t\star<\infty $ such that $ \lim\limits_{(x,t)\rightarrow(x\star,t\star) }|u(x,t)|=\infty $) by Navier-Stokes.

\end{document}