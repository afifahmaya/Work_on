\documentclass[a4paper,10pt]{article}
\usepackage[a4paper, hmargin={1.5cm,1.5cm}, vmargin={1.5cm,1.5cm}]{geometry}
\usepackage{amsmath}
\usepackage{amssymb}
\usepackage{amsthm}
\usepackage{amsfonts}
\usepackage{color}
\usepackage{graphicx}

\newtheorem{remark}{Remark}
\newtheorem{prop}{Proposition}
\newtheorem{theo}{Theorem}
\newtheorem{defi}{Definition}
\newtheorem{ex}{Example}
\newtheorem{note}{Notes}

\newcommand{\R}{\mathbb{R}}
\newcommand{\N}{\mathbb{N}}
\newcommand{\Z}{\mathbb{Z}}
\newcommand{\Q}{\mathbb{Q}}
\newcommand{\C}{\mathbb{C}}
\newcommand{\bx}{\mathbf{x}}
\newcommand{\by}{\mathbf{y}}

\begin{document}

\section{09-04-18}
We will learn about : Basics of functions of several variables.
In this lecture:

\subsection{A sequence in the Euclidean space and its application}
Using these notation :
\begin{itemize}
	\item $ \N $ : set of natural number $ (\N = \{1,2,3, \dots\} )$
	\item $ \Z $ : set of integers $ (\Z = \{0,\pm 1,\pm 2, \dots\} )$
	\item $ \Q $ : set of rational number $ (\Q = \{0,\pm 1, \pm 2, \dfrac{2}{3}, \dots\} )$
	\item $ \R $ : set of real number
	\item $ \C $ : set of complex number
\end{itemize}

\begin{defi}
	A sequence $ (x_{n})_{n=1}^{\infty} $ is an assignment of (real) number $ x_{n} \in \R $ to natural number $ n \in \N \ (x_{n} \in \R)$.\\
	Example : $ x_{n}=\dfrac{1}{n}. \ x_{1}=1,x_{2}=\dfrac{1}{2}, \dots $
\end{defi}

\begin{defi}
	A subsequence of a sequence $ (x_{n})_{n=1}^{\infty} $  is a sequence $ (y_{j})_{j=1}^{\infty} $ defined by $ y_{j}=x_{n_{j}} $ for some sequence $ (n_{j})_{j=1}^{\infty} $ in $ \N $ such that $ n_{j}<n_{j+1} \ (j=1,2,\dots) $.\\
	Example : sequence $ (x_{n})_{n=1}^{\infty} = 1, \dfrac{1}{2} , \dfrac{1}{3}, \dots, \dfrac{1}{100} $ , takes $ n_{1}=1, n_{2}=3, n_{3}=5, n_{4}=100 $\\
	subsequence $ (x_{n_{j}})_{j=1}^{\infty} = x_{n_{1}}, x_{n_{2}}, x_{n_{3}}, x_{n_{4}} = 1, \dfrac{1}{3}, \dfrac{1}{5}, \dfrac{1}{100}$.
\end{defi}

\begin{defi}
	Let $ (x_{n})_{n=1}^{\infty} $ be a sequence converges to $ \alpha \in \R $ if for any $ \epsilon>0 $, there is an $ N \in \N $ such that $ n>N , \ |x_{n}-\alpha|<\epsilon $.\\
	In the mathematical symbol $ \forall \epsilon > 0, \exists N \in \N $ such that $ n>N , \ |x_{n}-\alpha|<\epsilon $ for $ n>N $.\\
	In this case we write, $ \lim\limits_{n \rightarrow \infty} $ or $ x_{n}\rightarrow \alpha \ (n \rightarrow \infty ) $
\end{defi}

\begin{ex}
	
\end{ex}

\begin{theo}
	$ (x_{n})_{n=1}^{\infty} , (y_{n})_{n=1}^{\infty} $ is sequence. Suppose $ x_{n} \rightarrow \alpha $ and $ y_{n} \rightarrow \beta $ as $ n\rightarrow \infty $.
	\begin{enumerate}
		\item $ x_{n} \pm y_{n} \rightarrow \alpha \pm \beta , \ (n \rightarrow \infty) $
		\item $ x_{n} \cdot y_{n} \rightarrow \alpha \cdot \beta , \ (n \rightarrow \infty) $
		\item \label{theoseq3} if $ \beta \neq 0, \ \dfrac{x_{n}}{y_{n}} \rightarrow \dfrac{\alpha}{\beta} , \ (n \rightarrow \infty) $
	\end{enumerate}
\end{theo}

\begin{remark}
	On \ref{theoseq3}, $ \dfrac{x_{n}}{y_{n}} $ is not defined for all $ n \in \N $ because $ y_{n}=0 $ possibly for some $ n \in \N $. But, since $ y_{n} \rightarrow \beta \neq 0, \ y_{n} $ eventually is not 0. Hence $ \dfrac{x_{n}}{y_{n}} $ is defined eventually.
\end{remark}

\begin{theo}
	$ (x_{n})_{n=1}^{\infty} $ a sequence. If $ (x_{n})_{n=1}^{\infty} $ converges to $ \alpha \in \R $, any subsequence of $ (x_{n})_{n=1}^{\infty} $ converges to $ \alpha \in \R $.\\
	$ \because $ Let $ (x_{n})_{n=1}^{\infty} $ be a subsequence. Because $ x_{n} \rightarrow \alpha (n\rightarrow \infty), \forall\epsilon>0, \exists N \in \mathbb{N} $. Take $ J_{0} \in \mathbb{N} $ such that $ n_{j}>N_{\theta} $ for all $ y > J_{0} $. Then $ |x_{n_{j>N_{\theta}}}-\alpha|<\epsilon $ for $ j>J_{0} $. $ x_{n_{j}} \rightarrow \alpha (j \rightarrow \infty), (n_{j})_{j=1}^{\infty} $ also a sequence, $ n_{j} \in \mathbb{N}, n_{j}<n_{j+1} $.
\end{theo}

\textbf{Completeness Axiom.} Let $ (x_{n})_{n=1}^{\infty} $ be a monotonically increasing (decreasing) sequence (i.e. $ x_{n} \leq x_{n+1}, n\in \mathbb{N} $). Suppose that there is an $ M \in \R $ such that $ x_{n} \leq M (n \in \mathbb{N}) $ ($ x_{n} \geq M $). Then, $ (x_{n})_{n=1}^{\infty} $ converges ($ \exists \alpha\in\R $ such that $ x_{n}\rightarrow \alpha $)

\begin{theo}
	\textbf{Bolzano-Weirstrass.} $ (x_{n})_{n=1}^{\infty} $ a sequence in $ \R $. Suppose $ (x_{n})_{n=1}^{\infty} $ is bounded in the sense that $ |x_{n}|\leq M, \forall n \in \mathbb{N} $. Then $ (x_{n})_{n=1}^{\infty} $ contains a convergent subsequence.\\
	$ x_{n} $ is a peak of $ (x_{n})_{n=1}^{\infty} $ if $ x_{n} > x_{m} $ for $ m>n $.
\end{theo}

\newpage
\section{16-04-18}

\subsection{n-dimensional space}
$ \R^2 = \R \times \R \times \dots \times \R  = \{ (x_{1}, \dots, x_{n}) | x_{i}\in \R \} $.\\
Takes $ n=2 $, $ \R^2  \Leftrightarrow $ plane, we have $ P(a,b) $.\\
For $ n=3 $, we have $ P(a,b,c) $.

\begin{defi}
	$ P_{m} = (x_{1}^{m}, \dots, x_{n}^{m}) \in \R^n$, and $ \{ P_{m} \}_{m=1}^{\infty} $ :  a sequence in $ \R^{n} $.\\
	 $ \{P_{m}\} $ converges to $ A = (a_{1}, \dots , a_{n}) \in \R^n $, if $ \forall k=1, \dots, n , \ x_{k}^{m} \rightarrow a_{k} $ as $ n \rightarrow \infty $.
\end{defi}

\begin{defi}
	\textbf{Inner product and norm}.\\
	$ \bx = (x_{1}, \dots, x_{n}) , \by = (y_{1}, \dots, y_{n}  ) \in \R^n $. We can define : \\
	$ \bx \cdot \by = x_{1}y_{1} + \dots + x_{n}y_{n} $ ; \textbf{inner product}\\
	$ \| \bx \|= \sqrt{\bx \cdot \bx} $ ; \textbf{norm}
\end{defi}

\begin{ex}
	$ \bx \cdot \by =0 \Leftrightarrow \bx $ is perpendicular to $ \by $\\
	Takes $ n=0 $ then
	\begin{eqnarray} \nonumber
	x_{1}y_{1} + x_{2}y_{2} &=& 0 \\ \nonumber
	x_{1}y_{1} &=& -x_{2}y_{2} \\ \nonumber
	\dfrac{y_{1}}{y_{2}} &=& - \dfrac{x_{2}}{x_{1}}\\ \nonumber
	\text{ then } (x_{1},x_{2}) = c \cdot(-y_{2},y_{1})
	\end{eqnarray}
	pict : 
\end{ex}

\begin{ex}
	$ \|\bx\|=0 \Leftrightarrow x=0$\\
	$ (\Rightarrow) \ 0 = \| x \|^2 = x_{1}^2 + \dots + x_{n}^2 $, then $ x_{1}^2 =0 \ (\forall i =1, \dots, n )$ and finally $ x_{1}=0 $.
\end{ex}

\begin{note}
	%diganti note nanti
	$ \| x \| $ is the distance between $ \mathbf{0} = (0, \dots, 0) \in \R^n $ and $ \bx= (x_{1}, \dots, x_{n}) $.\\
	For notation, we will use $ P,Q \in \R^n $ as points and  $ \bx, \by \in \R^n $ as vectors. \\
	We also use $ \| x-y \| = \sqrt{(x_{1}-y_{1})^2 + \dots + (x_{n}-y_{n})^2} $ as distance between $ \bx $ and $ \by $. \\
	$ \| P-Q \| $ is distance between $ P $ and $ Q $.
	\[ \bx \pm \by = ( x_{1} \pm y_{1} , \dots, x_{n} \pm y_{n} ) \]
	\[ P=(p_{1}, \dots, p_{n}) ,  Q = (q_{1}, \dots, q_{n})  \text{, then } P+Q = (p_{1}+q_{1}, \dots, p_{n}+q_{n}) \]
	\[ \alpha \in \R, \ \alpha\bx = (\alpha x_{1}, \dots, \alpha x_{n}),  \alpha P = (\alpha p_{1}, \dots, \alpha p_{n})\]
	\[  \{P_{m}\}_{m=1}^{\infty} : \text{a sequence in } \R^n , \ P_{m} \rightarrow A \Leftrightarrow \| P_{m}-A \| \rightarrow 0 \]
\end{note}

\begin{theo}
	\textbf{Cauchy-Schwarz inequality}. For any $ \bx, \by \in \R^n $,
	\[ \mid \bx \cdot \by \mid \leq \| x \| \| y \| \]
	"$ = $" $ \Rightarrow  a\by = b\bx$ for some $ a,b \in \R $.\\
	$ \because $ We may assume $ \bx \neq \emptyset , \ \forall t \in \R $.
	\[ 0 \leq \| t\bx+\by \| = (t\bx+\by) (t\bx+\by) =  t^2 \|\bx\|^2 + 2t(\bx\cdot\by)+\|\by\|^2 \]
	\[D/4 \leq 0\]
\end{theo}

\begin{theo}
	\textbf{Bolzano=Weierstrass}. Let $ (P_{m})_{m=1}^{\infty} \subset \R^n $ be a sequence. Suppose that $ (P_{m})_{m=1}^{\infty} $ is bounded. In the sense that $ \|P_{m}\| \leq M (m \in \N) $ for some $ M \geq 0 $. Then $ (P_{m})_{m=1}^{\infty} $ contains a convergent subsequence.
\end{theo}

\begin{defi}
	\textbf{Ball}. $ A \in \R^n, R>0 $
	\[ \mathbf{B}(A,R) = \{ P \in \R^n | \|P-A\|<R \} \text{; open ball of center A with radius R} \]
	\[ \overline{\mathbf{B}}(A,R) = \{ P \in \R^n | \|P-A\|\leq R \} \text{; closed ball} \]
\end{defi}

\begin{defi}
	\begin{enumerate}
		\item $ E \subset \R^n $ is said to be \textbf{an open set} if $ E = \emptyset $ or $ \forall A \in E , \exists R>0 \text{ such that } \mathbf{B}(A,R) \subset E $.
		\item $ E \subset \R^n $ is said to be \textbf{a closed set} if $ E^{c} \in \R^n \ E $ is an open set.
	\end{enumerate}
	$ E : $ open, then neighbor in any point %tanya
\end{defi}

\begin{defi}
	\textbf{Accumulation point}. $ E \subset \R^n $; a set. $ A \in \R^n $ is called \textbf{an accumulation point} of $ E $ if $ \forall R>0, (\mathbf{B}(A,R)-\{A\}) \bigcap E \neq \emptyset $.
\end{defi}

\begin{note}
	$ E \subset \R^n $ is closed if and only if $ E $ contains any accumulation point of $ E $ . \textbf{Homework report, prove this}
\end{note}

\begin{note}
	\begin{enumerate}
		\item Both $ \emptyset $ and $ \R^n $ are open and closed
		\item $ \{E_{\lambda}\lambda \in A \} $; a collection of open sets $ \Rightarrow $ union $ \lambda \in A E_{\lambda} $ is also open 
		\item $ \{E_{\lambda}\}_{\lambda=1}^{N} $, a finite collection of open sets $ \Rightarrow $ irisan $ _{lamda=1}^{N} E_{\lambda} $ is also open.
		\item  \textbf{Rephrase of Bolzano Weierstrass theorem}. $ E \subset \R^n $ ; a \textbf{bounded closed set} $ \Leftrightarrow $ $ E $ is a closed set such that $ E \subset \mathbf{B}(\mathbf{0},R) $ for some $ R>0 $. $ E $ ; a bounded closed set then any sequence of $ E $ contains a convergent subsequence whose limit is in $ E $.
	\end{enumerate}
\end{note}

\begin{defi}
	A bounded closed set in $ \R^n $ is called \textbf{compact}.
\end{defi}

\begin{ex}
	$ \overline{\mathbf{B}}(A,R) $ is compact. \textbf{Report! prove this}
\end{ex}

\subsection{Continuity and differentiability of a function}

\subsubsection{Continuity}
$ E $ : a set in $ \R^n $ and $ f $ : is a function of $ E $ (real valued function).\\
 i.e. $ f $ is an assignment a (real) number to a point in $ E $. %insert pict

\begin{defi}
	\begin{enumerate}
		\item $ f $ is \textbf{continuous at $ A \in E $} if $ \forall (P_{m})_{m=1}^{\infty} \subset E $ : sequence with $ P_{m} \rightarrow A \ (m \rightarrow \infty) $
		\[ f(P_{m}) \rightarrow f(A) \ (m \rightarrow \infty) \]
		\item $ f $ is \textbf{continuous on E} if $ f $ is continuous at any point of $ E $.
	\end{enumerate}
\end{defi}

\subsubsection{Basic of continuous function on an interval in $ \R $}

\begin{theo}
	\textbf{Intermediate value theorem}. $ f $ : function on a closed interval $ [a,b] $ $ = \{ x \in \R | a \leq x \leq b \} $. Suppose that $ f(a) \leq f(b) $. Then, $ \forall \gamma $ with $ f(a) \leq \gamma \leq f(b) $, $ \exists c \in [a,b] $ with $ f(c)=\gamma $.
\end{theo}

\begin{theo}
	\textbf{Extreme value theorem}. $ f $ is a continuous function on a closed interval $ [a,b] $. Then, $ f $ attains a maximum and a minimum on $ [a,b] $.
\end{theo}

\newpage
 Name	: Afifah Maya Iknaningrum  (1715011053)\\ \\
 \hspace{2cm}
\textbf{REPORT 1}

\begin{enumerate}
	\item $ E \subset \R^n $ is closed if and only if $ E $ contains any accumulation point of $ E $
	\item $ \overline{\mathbf{B}}(A,R) $ is compact.
\end{enumerate}

\textbf{Proof :}
\begin{enumerate}
	\item $ (\Rightarrow) $ if $ E $ is closed then $ E $ contains all of its accumulation point. Let $ x $ accumulation point of $ E, \ x \in E $ and $ E $ is closed then $ E^{c} $ is open . \\ \\ 
	Let $ x \in E^{c} $ and $ R>0 \Rightarrow \forall x\in E^{c}, \exists B(x,R) $ such that $ \forall y \in B(x,R) \Rightarrow y \in E^{c} $.\\
	Suppose $ x $ is accumulation point of $ E $ that is not in $ E $.\\
	 Then, $ \forall e \in B(x,R), \exists y \neq x $ with $ y\in e \bigcap E $. \\
	$ y\in e\bigcap E \Rightarrow y \notin E^{c} $ contradiction.\\ \\
	$ (\Leftarrow) $ $ E $ contains all of its accumulation point then $ E $ is closed. \\ \\
	Suppose $ E $ contains all of its accumulation point. Suppose $ E^{c} $ is not open. \\
	$ \exists x \in E^{c} $ such that $ \forall e \in B(x,R), R>0, \exists y \in e $ that also in $ E $.\\
	Its contradict the premise, because $ x $ is accumulation point.
	\item Suppose $ x \notin \overline{B}(A,R) \Rightarrow \|x-A\| >R $.\\
	So let $ \|x-A\| -R = \epsilon >0 $. \\
	Consider $ y \in B(x, \epsilon/2) $,
	\begin{eqnarray} \nonumber
	\|y-A\| &\geq& \|x-A\| -\|y-x\| \\ \nonumber
	\|y-A\| &\geq& R + \epsilon - (\epsilon/2) \\ \nonumber
	\|y-A\| &\geq& R + (\epsilon/2) \\ \nonumber
	\|y-A\| &>& R
	\end{eqnarray}
	shows that $ y \in \overline{B}(A,R) $. Hence $ B(x, \epsilon/2) $ subset of $ \overline{B}(A,R)^{c} $.\\
	Because $ \overline{B}(A,R)^{c} $ hence $ \overline{B}(A,R) $ is closed. \\ \\
	By definition, $ \overline{B}(A,R) = \{ x\in \R^n | \|x-A\| \leq R \}$\\
	Then $ \forall x \in \overline{B}(A,R) $ we can find
	\begin{eqnarray}\nonumber
	&\|x-A\|& \leq R\\
	-R \leq&|x-A|& \leq R.
	\end{eqnarray}
	shows that $ \overline{B}(A,R) $ is bounded.\\ \\
	Because closed and bounded, $ \overline{B}(A,R) $ is compact.
\end{enumerate}

\newpage
\section{23-04-18}

$ f $ is \underline{continuous function} on $ [a,b] = \{x\in\R | a \leq x \leq b \} $

\begin{theo}
	\textbf{Intermediate Value Theorem}. Suppose $ f(a) \leq f(b) $ then $ \forall \gamma \in \R $ with $ f(a) \leq \gamma \leq f(b), \ \exists c\ in [a,b] $ such that $ f(c) = \gamma $.\hspace{3cm}
\end{theo}

\begin{theo}
	\textbf{Extreme value theorem}. $ f $ attains a maximum and a minimum on $ [a,b] $. \hspace{3cm}
\end{theo}

\subsection{Differentiable function on intervals}

$ f : $ function defined around $ x=a\in \R $.\\
$ f $ is \underline{differentiable at $ x=a $} if the limit $ f \prime (a) = \lim\limits_{h\rightarrow 0} \dfrac{f(a+h)-f(a)}{h}$ exist.\\ \\
$ f : $ function on $ (a,b)=\{ x\in\R | a<x<b \} $\\
$ f $ is \underline{differentiable on $ (a,b) $} if $ f $ is differentiable at any point of $ (a,b) $.\\ \\
\textbf{Properties :} if $ f $ is differentiable at $ x=a $ then $ f $ is continuous at $ x=a $.\\
$ \because f(a+h)=f(a) +h \dfrac{f(a+h)-f(a)}{h} = f(a)+hf\prime(a) $. Because $ h \rightarrow 0 $ then $ f(a+h)\rightarrow f(a) $.\\

\begin{theo}
	\textbf{Rolle's theorem}. $ f :$ continue on $ [a,b] $ and differentiable on $ (a,b) $. If $ f(a)=f(b) $ then $ \exists c \in (a,b) $ such that $ f\prime (c)=0 $.\\
	$ \because $ if $ f $is a constant function, $ f\prime(x)=0, \forall x\in (a,b) $. Suppose that $ f $ is not a constant function, by \underline{extreme value theorem}, $ f $ attain $ max $ at $ x=c_{1} $ and $ min $ at $ x=c_{2} $ with $ c_{1} \neq c_{2} $. \\
	(Otherwise $ max=f(c_{1}) = f(c_{2}) = min $)\\ \\
	$ \forall x\in [a,b], f(c_{2}) \leq min \leq f(x) \leq max \leq f(a) $. We may assume $ c_{1} \in (a,b) $.\\
	(Otherwise, consider $ -f $ instead $ f $; $ (-f)\prime(a) =\lim\limits_{h\rightarrow 0} \dfrac{-f(a+h)-(-f(a))}{h} = -f\prime(a) $ )\\ \\
	\begin{eqnarray}\nonumber
	\dfrac{f(c_{1}+h)-f(c_{1})}{h} \leq 0, h<0. & \text{ for } h \rightarrow 0 ,& f\prime(c_{1}) \leq 0 \\ \nonumber
	\dfrac{f(c_{1}+h)-f(c_{1})}{h} \geq 0, h>0. & \text{ for } h \rightarrow 0 ,& f\prime(c_{1}) \geq 0
	\end{eqnarray}
	Then, we can conclude that $ f\prime(c_{1}) = 0 $
\end{theo}

\begin{theo}
	\textbf{Meanvalue theorem}. $ f : $ continuous on $ [a,b] $ and differntiable on $ (a,b) $.\\
	$ \exists c \in (a,b) $ such that $ \dfrac{f(b)-f(a)}{b-a} = f\prime (c) $.\\
	$ \because $ consider $ F(x) = f(x) - f(a) - \dfrac{f(b)-f(a)}{b-a}(x-a) $ and apply \underline{the Rolle's theorem}.
\end{theo}

\subsection{Basic of function of several variables}
$ D \subset \R^n $ is a domain $ \Leftrightarrow D $ is open. Any two points of $ D $ are connected by a polygonal arc in $ D $. We can consider a ball $ \textbf{B}(P,R) $.\\
\textbf{Note :} From now on, we discuss with $ \R^2 $ for simplicity.

\subsubsection{The partial derivative at $ P(a,b) $}
1D : $ f\prime(a) = \lim\limits_{h\rightarrow 0} \dfrac{f(a+h)-f(a)}{h} $.\\
2D : $ f_{x}(a,b) = \dfrac{\partial f}{\partial x} (a,b) = \lim\limits_{h\rightarrow 0} \dfrac{f(a+h,b)-f(a,b)}{h} $.\\
$ f_{y}(a,b) = \dfrac{\partial f}{\partial y} (a,b) = \lim\limits_{h\rightarrow 0} \dfrac{f(a,b+h)-f(a,b)}{h} $

\begin{defi}
	$ f $ is \underline{partially differentiable at $ P(a,b) $} if $ f_{x}(a,b), f_{y}(a,b) $ exist. And $ f $ is \underline{partially differentiable on $ D $} if $ f $ is partially differentiable at any point on $ D $.
\end{defi}

\subsubsection{Landau symbol}
$ O : $ big $ o $ and $ o : $ small $ o $ describe the behavior of function.

\end{document}