\documentclass[a4paper,10pt]{article}
\usepackage[a4paper, hmargin={1.5cm,1.5cm}, vmargin={1.5cm,1.5cm}]{geometry}
\usepackage{amsmath}
\usepackage{amssymb}
\usepackage{amsthm}
\usepackage{amsfonts}
\usepackage{color}
\usepackage{graphicx}

\newtheorem{remark}{Remark}
\newtheorem{prop}{Proposition}
\newtheorem{theo}{Theorem}
\newtheorem{defi}{Definition}
\newtheorem{ex}{Example}
\newtheorem{note}{Notes}

\begin{document}

\section{09-04-2018}
We will learn about : Basics of functions of several variables.
In this lecture:

\subsection{A sequence in the Euclidean space and its application}
Using these notation :
\begin{itemize}
	\item $ \mathbb{N} $ : set of natural number $ (\mathbb{N} = \{1,2,3, \dots\} )$
	\item $ \mathbb{Z} $ : set of integers $ (\mathbb{Z} = \{0,\pm 1,\pm 2, \dots\} )$
	\item $ \mathbb{Q} $ : set of rational number $ (\mathbb{Q} = \{0,\pm 1, \pm 2, \dfrac{2}{3}, \dots\} )$
	\item $ \mathbb{R} $ : set of real number
	\item $ \mathbb{C} $ : set of complex number
\end{itemize}

\begin{defi}
	A sequence $ (x_{n})_{n=1}^{\infty} $ is an assignment of (real) number $ x_{n} \in \mathbb{R} $ to natural number $ n \in \mathbb{N} \ (x_{n} \in \mathbb{R})$.\\
	Example : $ x_{n}=\dfrac{1}{n}. \ x_{1}=1,x_{2}=\dfrac{1}{2}, \dots $
\end{defi}

\begin{defi}
	A subsequence of a sequence $ (x_{n})_{n=1}^{\infty} $  is a sequence $ (y_{j})_{j=1}^{\infty} $ defined by $ y_{j}=x_{n_{j}} $ for some sequence $ (n_{j})_{j=1}^{\infty} $ in $ \mathbb{N} $ such that $ n_{j}<n_{j+1} \ (j=1,2,\dots) $.\\
	Example : sequence $ (x_{n})_{n=1}^{\infty} = 1, \dfrac{1}{2} , \dfrac{1}{3}, \dots, \dfrac{1}{100} $ , takes $ n_{1}=1, n_{2}=3, n_{3}=5, n_{4}=100 $\\
	subsequence $ (x_{n_{j}})_{j=1}^{\infty} = x_{n_{1}}, x_{n_{2}}, x_{n_{3}}, x_{n_{4}} = 1, \dfrac{1}{3}, \dfrac{1}{5}, \dfrac{1}{100}$.
\end{defi}

\begin{defi}
	Let $ (x_{n})_{n=1}^{\infty} $ be a sequence converges to $ \alpha \in \mathbb{R} $ if for any $ \epsilon>0 $, there is an $ N \in \mathbb{N} $ such that $ n>N , \ |x_{n}-\alpha|<\epsilon $.\\
	In the mathematical symbol $ \forall \epsilon > 0, \exists N \in \mathbb{N} $ such that $ n>N , \ |x_{n}-\alpha|<\epsilon $ for $ n>N $.\\
	In this case we write, $ \lim\limits_{n \rightarrow \infty} $ or $ x_{n}\rightarrow \alpha \ (n \rightarrow \infty ) $
\end{defi}

\begin{ex}
	
\end{ex}

\begin{theo}
	$ (x_{n})_{n=1}^{\infty} , (y_{n})_{n=1}^{\infty} $ is sequence. Suppose $ x_{n} \rightarrow \alpha $ and $ y_{n} \rightarrow \beta $ as $ n\rightarrow \infty $.
	\begin{enumerate}
		\item $ x_{n} \pm y_{n} \rightarrow \alpha \pm \beta , \ (n \rightarrow \infty) $
		\item $ x_{n} \cdot y_{n} \rightarrow \alpha \cdot \beta , \ (n \rightarrow \infty) $
		\item \label{theoseq3} if $ \beta \neq 0, \ \dfrac{x_{n}}{y_{n}} \rightarrow \dfrac{\alpha}{\beta} , \ (n \rightarrow \infty) $
	\end{enumerate}
\end{theo}

\begin{remark}
	On \ref{theoseq3}, $ \dfrac{x_{n}}{y_{n}} $ is not defined for all $ n \in \mathbb{N} $ because $ y_{n}=0 $ possibly for some $ n \in \mathbb{N} $. But, since $ y_{n} \rightarrow \beta \neq 0, \ y_{n} $ eventually is not 0. Hence $ \dfrac{x_{n}}{y_{n}} $ is defined eventually.
\end{remark}

\begin{theo}
	$ (x_{n})_{n=1}^{\infty} $ a sequence. If $ (x_{n})_{n=1}^{\infty} $ converges to $ \alpha \in \mathbb{R} $, any subsequence of $ (x_{n}) $
\end{theo}

\section{16-04-2018}

\subsection{n-dimensional space}
$ \mathbb{R}^2 = \mathbb{R} \times \mathbb{R} \times \dots \times \mathbb{R}  = \{ (x_{1}, \dots, x_{n}) | x_{i}\in \mathbb{R} \} $.\\
Takes $ n=2 $, $ \mathbb{R}^2  \Leftrightarrow $ plane, we have $ P(a,b) $.\\
For $ n=3 $, we have $ P(a,b,c) $.

\begin{defi}
	$ P_{m} = (x_{1}^{m}, \dots, x_{n}^{m}) \in \mathbb{R}^n$, and $ \{ P_{m} \}_{m=1}^{\infty} $ :  a sequence in $ \mathbb{R}^{n} $.\\
	 $ \{P_{m}\} $ converges to $ A = (a_{1}, \dots , a_{n}) \in \mathbb{R}^n $, if $ \forall k=1, \dots, n , \ x_{k}^{m} \rightarrow a_{k} $ as $ n \rightarrow \infty $.
\end{defi}

\begin{defi}
	\textbf{Inner product and norm}.\\
	$ \mathbf{x} = (x_{1}, \dots, x_{n}) , \mathbf{y} = (y_{1}, \dots, y_{n}  ) \in \mathbb{R}^n $. We can define : \\
	$ \mathbf{x} \cdot \mathbf{y} = x_{1}y_{1} + \dots + x_{n}y_{n} $ ; \textbf{inner product}\\
	$ || \mathbf{x} ||= \sqrt{\mathbf{x} \cdot \mathbf{x}} $ ; \textbf{norm}
\end{defi}

\begin{ex}
	$ \mathbf{x} \cdot \mathbf{y} =0 \Leftrightarrow \mathbf{x} $ is perpendicular to $ \mathbf{y} $\\
	Takes $ n=0 $ then
	\begin{eqnarray} \nonumber
	x_{1}y_{1} + x_{2}y_{2} &=& 0 \\ \nonumber
	x_{1}y_{1} &=& -x_{2}y_{2} \\ \nonumber
	\dfrac{y_{1}}{y_{2}} &=& - \dfrac{x_{2}}{x_{1}}\\ \nonumber
	\text{ then } (x_{1},x_{2}) = c \cdot(-y_{2},y_{1})
	\end{eqnarray}
	pict : 
\end{ex}

\begin{ex}
	$ ||\mathbf{x}||=0 \Leftrightarrow x=0$\\
	$ (\Rightarrow) \ 0 = || x ||^2 = x_{1}^2 + \dots + x_{n}^2 $, then $ x_{1}^2 =0 \ (\forall i =1, \dots, n )$ and finally $ x_{1}=0 $.
\end{ex}

\begin{note}
	%diganti note nanti
	$ || x || $ is the distance between $ \mathbf{0} = (0, \dots, 0) \in \mathbb{R}^n $ and $ \mathbf{x}= (x_{1}, \dots, x_{n}) $.\\
	For notation, we will use $ P,Q \in \mathbb{R}^n $ as points and  $ \mathbf{x}, \mathbf{y} \in \mathbb{R}^n $ as vectors. \\
	We also use $ || x-y || = \sqrt{(x_{1}-y_{1})^2 + \dots + (x_{n}-y_{n})^2} $ as distance between $ \mathbf{x} $ and $ \mathbf{y} $. \\
	$ || P-Q || $ is distance between $ P $ and $ Q $.
	\[ \mathbf{x} \pm \mathbf{y} = ( x_{1} \pm y_{1} , \dots, x_{n} \pm y_{n} ) \]
	\[ P=(p_{1}, \dots, p_{n}) ,  Q = (q_{1}, \dots, q_{n})  \text{, then } P+Q = (p_{1}+q_{1}, \dots, p_{n}+q_{n}) \]
	\[ \alpha \in \mathbb{R}, \ \alpha\mathbf{x} = (\alpha x_{1}, \dots, \alpha x_{n}),  \alpha P = (\alpha p_{1}, \dots, \alpha p_{n})\]
	\[  \{P_{m}\}_{m=1}^{\infty} : \text{a sequence in } \mathbb{R}^n , \ P_{m} \rightarrow A \Leftrightarrow || P_{m}-A || \rightarrow 0 \]
\end{note}

\begin{theo}
	\textbf{Cauchy-Schwarz inequality}. For any $ \mathbf{x}, \mathbf{y} \in \mathbb{R}^n $,
	\[ \mid \mathbf{x} \cdot \mathbf{y} \mid \leq || x || || y || \]
	"$ = $" $ \Rightarrow  a\mathbf{y} = b\mathbf{x}$ for some $ a,b \in \mathbb{R} $.\\
	$ \therefore $ We may assume $ \mathbf{x} \neq \emptyset , \ \forall t \in \mathbb{R} $.
	\[ 0 \leq || t\mathbf{x}+\mathbf{y} || = (t\mathbf{x}+\mathbf{y}) (t\mathbf{x}+\mathbf{y}) =  t^2 ||\mathbf{x}||^2 + 2t(\mathbf{x}\cdot\mathbf{y})+||\mathbf{y}||^2 \]
	\[D/4 \leq 0\]
\end{theo}

\begin{theo}
	\textbf{Bolzano=Weierstrass}. Let $ (P_{m})_{m=1}^{\infty} \subset \mathbb{R}^n $ be a sequence. Suppose that $ (P_{m})_{m=1}^{\infty} $ is bounded. In the sense that $ ||P_{m}|| \leq M (m \in \mathbb{N}) $ for some $ M \geq 0 $. Then $ (P_{m})_{m=1}^{\infty} $ contains a convergent subsequence.
\end{theo}

\begin{defi}
	\textbf{Ball}. $ A \in \mathbb{R}^n, R>0 $
	\[ \mathbf{B}(A,R) = \{ P \in \mathbb{R}^n | ||P-A||<R \} \text{; open ball of center A with radius R} \]
	\[ \bar{\mathbf{B}}(A,R) = \{ P \in \mathbb{R}^n | ||P-A||\leq R \} \text{; closed ball} \]
\end{defi}

\begin{defi}
	\begin{enumerate}
		\item $ E \subset \mathbb{R}^n $ is said to be \textbf{an open set} if $ E = \emptyset $ or $ \forall A \in E , \exists R>0 \text{ such that } \mathbf{B}(A,R) \subset E $.
		\item $ E \subset \mathbb{R}^n $ is said to be \textbf{a closed set} if $ E^{c} = \mathbb{R}^n \ E $ is an open set.
	\end{enumerate}
	$ E : $ open, then neighbor in any point %tanya
\end{defi}

\begin{defi}
	\textbf{Accumulation point}. $ E \subset \mathbb{R}^n $; a set. $ A \in \mathbb{R}^n $ is called \textbf{an accumulation point} of $ E $ if $ \forall R>0, (\mathbf{B}(A,R)-\{A\}) \text{irisan } E \neq \emptyset $.
\end{defi}

\begin{remark}
	ini notes. $ E \subset \mathbb{R}^n $ is closed if and only id $ E $ contains any accumulation point of $ E $ . \textbf{Homework report, prove this}
\end{remark}

\begin{remark}
	note juga.
	\begin{enumerate}
		\item Both $ \emptyset $ and $ \mathbb{R}^n $ are open and closed
		\item $ \{E_{\lambda}\lambda \in A \} $; a collection of open sets $ \Rightarrow $ union $ \lambda \in A E_{\lambda} $ is also open 
		\item $ \{E_{\lambda}\}_{\lambda=1}^{N} $, a finite collection of open sets $ \Rightarrow $ irisan $ _{lamda=1}^{N} E_{\lambda} $ is also open.
		\item  \textbf{Rephrase of Bolzano Weierstrass theorem}. $ E \subset \mathbb{R}^n $ ; a \textbf{ounded closed set} $ \Leftrightarrow $ $ E $ is a closed set such that $ E \subset \mathbf{B}(\mathbb{0},R) $ for some $ R>0 $. $ E $ ; a bounded closed set then any sequence of $ E $ contains a convergent subsequence whose limit is in $ E $.
	\end{enumerate}
\end{remark}

\begin{defi}
	A bounded closed set in $ \mathbb{R}^n $ is called \textbf{compact}.
\end{defi}

\begin{ex}
	$ \bar{\mathbf{B}}(A,R) $ is compact. \textbf{Report! prove this}
\end{ex}

\subsection{Continuity and differentiability of a function}

\subsubsection{Continuity}
$ E $ : a set in $ \mathbb{R}^n $ and $ f $ : is a function of $ E $ (real valued function). i.e. $ f $ is an assignment a (real) number to a point in $ E $. %insert pict

\begin{defi}
	\begin{enumerate}
		\item $ f $ is \textbf{continuous at $ A \in E $} if $ \forall (P_{m})_{m=1}^{\infty} \subset E $ : sequence with $ P_{m} \rightarrow A \ (m \rightarrow \infty) $
		\[ f(P_{m}) \rightarrow f(A) \ (m \rightarrow \infty) \]
		\item $ f $ is \textbf{continuous on E} if $ f $ is continuous at any point of $ E $.
	\end{enumerate}
\end{defi}

\subsubsection{Basic of continuous function on an interval in $ \mathbb{R} $}

\begin{theo}
	\textbf{Intermediate value theorem}. $ f $ : function on a closed interval $ [a,b] $ $ = \{ x \in \mathbb{R} | a \leq x \leq b \} $. Suppose that $ f(a) \leq f(b) $. Then, $ \forall \gamma $ with $ f(a) \leq \gamma \leq f(b) $, $ \exists c \in [a,b] $ with $ f(c)=\gamma $.
\end{theo}

\begin{theo}
	\textbf{Extreme value theorem}. $ f $ is a continuous function
\end{theo}

\end{document}