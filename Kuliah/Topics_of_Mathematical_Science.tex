\documentclass[a4paper,10pt]{article}
\usepackage[a4paper, hmargin={1.5cm,1.5cm}, vmargin={1.5cm,1.5cm}]{geometry}
\usepackage{amsmath}
\usepackage{amsthm}
\usepackage{amsfonts}
\usepackage{color}
\usepackage{graphicx}

\newtheorem{remark}{Remark}[]
\newtheorem{prop}{Proposition}
\newtheorem{theo}{Theorem}[]
\newtheorem{defi}{Definition}[]
\newtheorem{ex}{Example}[]

\begin{document}

\section{09-04-2018}
We will learn about : Basics of functions of several variables.
In this lecture:

\subsection{A sequence in the Euclidean space and its application}
Using these notation :
\begin{itemize}
	\item $ \mathbb{N} $ : set of natural number $ (\mathbb{N} = \{1,2,3, \dots\} )$
	\item $ \mathbb{Z} $ : set of integers $ (\mathbb{Z} = \{0,\pm 1,\pm 2, \dots\} )$
	\item $ \mathbb{Q} $ : set of rational number $ (\mathbb{Q} = \{0,\pm 1, \pm 2, \dfrac{2}{3}, \dots\} )$
	\item $ \mathbb{R} $ : set of real number
	\item $ \mathbb{C} $ : set of complex number
\end{itemize}

\begin{defi}
	A sequence $ (x_{n})_{n=1}^{\infty} $ is an assignment of (real) number $ x_{n} \in \mathbb{R} $ to natural number $ n \in \mathbb{N} \ (x_{n} \in \mathbb{R})$.\\
	Example : $ x_{n}=\dfrac{1}{n}. \ x_{1}=1,x_{2}=\dfrac{1}{2}, \dots $
\end{defi}

\begin{defi}
	A subsequence of a sequence $ (x_{n})_{n=1}^{\infty} $  is a sequence $ (y_{j})_{j=1}^{\infty} $ defined by $ y_{j}=x_{n_{j}} $ for some sequence $ (n_{j})_{j=1}^{\infty} $ in $ \mathbb{N} $ such that $ n_{j}<n_{j+1} \ (j=1,2,\dots) $.\\
	Example : sequence $ (x_{n})_{n=1}^{\infty} = 1, \dfrac{1}{2} , \dfrac{1}{3}, \dots, \dfrac{1}{100} $ , takes $ n_{1}=1, n_{2}=3, n_{3}=5, n_{4}=100 $\\
	subsequence $ (x_{n_{j}})_{j=1}^{\infty} = x_{n_{1}}, x_{n_{2}}, x_{n_{3}}, x_{n_{4}} = 1, \dfrac{1}{3}, \dfrac{1}{5}, \dfrac{1}{100}$.
\end{defi}

\begin{defi}
	Let $ (x_{n})_{n=1}^{\infty} $ be a sequence converges to $ \alpha \in \mathbb{R} $ if for any $ \epsilon>0 $, there is an $ N \in \mathbb{N} $ such that $ n>N , \ |x_{n}-\alpha|<\epsilon $.\\
	In the mathematical symbol $ \forall \epsilon > 0, \exists N \in \mathbb{N} $ such that $ n>N , \ |x_{n}-\alpha|<\epsilon $ for $ n>N $.\\
	In this case we write, $ \lim\limits_{n \rightarrow \infty} $ or $ x_{n}\rightarrow \alpha \ (n \rightarrow \infty ) $
\end{defi}

\begin{ex}
	
\end{ex}

\begin{theo}
	$ (x_{n})_{n=1}^{\infty} , (y_{n})_{n=1}^{\infty} $ is sequence. Suppose $ x_{n} \rightarrow \alpha $ and $ y_{n} \rightarrow \beta $ as $ n\rightarrow \infty $.
	\begin{enumerate}
		\item $ x_{n} \pm y_{n} \rightarrow \alpha \pm \beta , \ (n \rightarrow \infty) $
		\item $ x_{n} \cdot y_{n} \rightarrow \alpha \cdot \beta , \ (n \rightarrow \infty) $
		\item \label{theoseq3} if $ \beta \neq 0, \ \dfrac{x_{n}}{y_{n}} \rightarrow \dfrac{\alpha}{\beta} , \ (n \rightarrow \infty) $
	\end{enumerate}
\end{theo}

\begin{remark}
	On \ref{theoseq3}, $ \dfrac{x_{n}}{y_{n}} $ is not defined for all $ n \in \mathbb{N} $ because $ y_{n}=0 $ possibly for some $ n \in \mathbb{N} $. But, since $ y_{n} \rightarrow \beta \neq 0, \ y_{n} $ eventually is not 0. Hence $ \dfrac{x_{n}}{y_{n}} $ is defined eventually.
\end{remark}

\begin{theo}
	$ (x_{n})_{n=1}^{\infty} $ a sequence. If $ (x_{n})_{n=1}^{\infty} $ converges to $ \alpha \in \mathbb{R} $, any subsequence of $ (x_{n}) $
\end{theo}

\end{document}