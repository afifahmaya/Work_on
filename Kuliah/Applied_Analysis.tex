\documentclass[a4paper,10pt]{article}
\usepackage[a4paper, hmargin={1.5cm,1.5cm}, vmargin={1.5cm,1.5cm}]{geometry}
\usepackage{amsmath}
\usepackage{amsthm}
\usepackage{amsfonts}
\usepackage{color}
\usepackage{graphicx}

\newtheorem{remark}{Remark}
\newtheorem{prop}{Properties}
\newtheorem{theo}{Theorem}
\newtheorem{defi}{Definition}
\newtheorem{ex}{Example}

\newcommand{\R}{\mathbb{R}}

\begin{document}

\section{20-04-2018}

\subsection{Membranes}
\begin{remark}
	Consider the problem below with $ \phi : \Omega \rightarrow \R $ unknown and given $ f : \Omega \rightarrow \R $
	\begin{equation}\label{2.2}
			\begin{cases}
			-\bigtriangleup \phi = f &, \text{ in } \Omega \\
			\dfrac{\partial \phi}{\partial n} = 0 &,\text{ on } \Gamma \equiv \partial \Omega 
			\end{cases}
	\end{equation}
	\begin{enumerate}
		\item If $ \phi $ is the solution of (\ref{2.2}), then $ \phi + c $ will be also a solution for $ c \in \R $. So, there is no uniqueness of (\ref{2.2}).
		\item Condition for $ f $
		\[ \int_{\Omega} f \ dx = \int_{\Omega} - \bigtriangleup \phi \ dx = \int_{\Omega} \nabla \cdot (-\nabla \phi) \ dx =  \]
	\end{enumerate}
\end{remark}

\end{document}