\documentclass[a4paper,10pt]{article}
\usepackage[a4paper, hmargin={1.5cm,1.5cm}, vmargin={1.5cm,1.5cm}]{geometry}
\usepackage{amsmath}
\usepackage{amsthm}
\usepackage{amsfonts}
\usepackage{color}
\usepackage{graphicx}

\newtheorem{remark}{Remark}
\newtheorem{prop}{Properties}
\newtheorem{theo}{Theorem}
\newtheorem{defi}{Definition}
\newtheorem{ex}{Example}

\begin{document}

\section{16-04-2018}

\subsection{Definitions and basic properties of polynomial}
$ \mathbb{N} $ = set of natural number\\
$ \mathbb{N}_{0}  = \mathbb{N} union \{0\}$\\
for $ n \in \mathbb{N} $, $ \mathbb{N}_{0}^{n} = \{ \alpha=(\alpha_{1}, \dots, \alpha_{n}) | \alpha_{1}, \dots, \alpha_{n} \in \mathbb{N}_{0} \} $ is semi-module because closed over addition (?)\\
$ \mathbb{0} = (0, \dots, 0) $ and $ x_{1}, \dots, x_{n} $ ; variables\\
for $ \alpha=(\alpha_{1}, \dots, \alpha_{n}) \in \mathbb{N}_{0}^{n} $\\
\textbf{a mononomial}, or direct product of variables $ x^{\alpha} = \begin{cases}
1 , (if \alpha=0) \\ x_{1}^{\alpha_{1}} \ x_{2}^{\alpha_{2}} \ \dots x_{n}^{\alpha_{n}}, (otherwise)
\end{cases} $

$ K $ is field. [Field : is a set on which addition, subtraction, multiplication, and division are defined, and behave as when they are applied to rational and real numbers.]

\begin{defi}
	Let $ A \subset \mathbb{N}_{0}^{n} $ :finite
	\[ f = \sum_{\alpha \in A} c_{\alpha} x^{\alpha} \ (c_{\alpha} \in K) \]
	is called \textbf{a polynomial} of $ x_{1}, \dots, x_{n} $ with K-coefficients.\\
	\[ K[\mathbf{x}] = K [x_{1}, \dots, x_{n}]  = \{ f | f \text{is a polynomial of } x_{1}, \dots, x_{n}  \text{ with K-coefficients }\}\]
	\[M_{n} = \{ x^{\alpha} | \alpha \in \mathbb{N}_{0}^{n} \} \subset K[x] \]
\end{defi}

\begin{ex}
	$ n=2 $ then we have $ A =\{ (0,0), (1,1), (0,3), (2,0), (2,1) \} $.
	\[ f = x_{1}^2 x_{2} + 5x_{2}^3 -2x_{1}x_{2} + 10 \]
	$ C_{(2,1)} =1 , C_{(2,0)}=0, C_{(0,3)}=5, C_{(1,1)}=-2, C_{(0,0)}=10 $.
\end{ex}

\begin{defi}
	\textbf{Support}. $ f = \sum_{\alpha \in A} c_{\alpha} x^{\alpha} \neq 0$ then
	\[ supp(f) = \{ \alpha \in A | C_{\alpha} \neq 0 \} \]
\end{defi}

\begin{ex}
	$ supp(f) = \{ (0,0), (1,1), (0,3), (2,1) \} $
\end{ex}

\begin{defi}
	\textbf{Total degree}. $ |\alpha| = \alpha_{1} + \dots + \alpha_{n} (\alpha \in \mathbb(N)_{0}^{n} )$. If $ supp(f) \neq \emptyset $
	\[ tdeg(f) = max \{ |\alpha| | \alpha \in supp(f) \} \]
\end{defi}

\begin{ex}
	$ tdeg(f) = max \{ 0,2,3,3 \} =3 $
\end{ex}


$ f,g \in K[x] $\\
$ f~g $ or associated $ \Leftrightarrow \exists C \in K \ \{0\} $ such that $ f= c \cdot g $. \\
For example : $ f = x_{1}^2 x_{2} +1 ; g = 3x_{1}^2 x_{2} +3 ; h= 3x_{1}^2 x_{2} +2 $. Then $ f~g, f not~ h $

$ f|g $ or $ f $ devides $ g $ $ \Leftrightarrow \exists h \in K[x] $ such that $ f \cdot h = g $

\begin{prop}
	$ f | g \Rightarrow tdeg(f) \leq tdeg(g) $
\end{prop}

\begin{defi}
	Let $ f \in K[x] \ K $. $ f $ is \textbf{ irreducible } if $ \Big( h|f \Rightarrow (h \in K or h~f) \Big) $. If $ tdeg(f) > 0 $ and $ f $ is not irreducible, then $ f $ is called \textbf{reducible}.
\end{defi}

\begin{theo}
	Let $ f \in K[x] \ K $. Then $ f $ can be \textbf{factorized} as
	\begin{enumerate}
		\item $ f = c \ g_{1}^{\beta_{1}} \ g_{2}^{\beta_{2}} \dots g_{n}^{\beta_{m}} $ where $ c \in K \ \{0\} $, $ \beta_{1}, \beta_{2}, \dots, \beta_{m} \in \mathbb{N}$, and $ g_{1}, \dots, g_{m} $ : irreducible, $ g_{i} not ~ g_{j} \ (i \neq j)$
		\item if $ f = c \ g_{1}^{\beta_{1}} \ g_{2}^{\beta_{2}} \dots g_{m}^{\beta_{m}} = d \ h_{1}^{\gamma_{1}} \ h_{2}^{\gamma_{2}} \dots h_{l}^{\gamma_{l}}$ (factorization). Then (a) $ m=l $, (b) by change of index, $ g_{1}~h_{1}, \dots, g_{m}~h_{m} $.
	\end{enumerate}
\end{theo}

We can define $ GCD(f,g) $ for $ f,g \in K[x] , \big( (f,g)\neq(0,0) \big)$ 

\begin{defi}
	Let $ I \in K[x] , I \neq \emptyset $. $ I $ is \textbf{an ideal} if
	\begin{enumerate}
		\item $ f,g \in I \Rightarrow f+g \in I $
		\item $ f \in I , r \in K[x] \Rightarrow r \cdot f \in I $
	\end{enumerate}
\end{defi}

\begin{defi}
	\textbf{An ideal generated by $ f_{1}, \dots, f_{m} $}. Let $ f_{1}, \dots, f_{m} \in K[x] \ \{0\}$
	\[ < f_{1}, \dots, f_{m} > = \{ r_{1}f_{1} + r_{2}f_{2} + \dots + r_{m}f_{m} | r_{1}, r_{2}, \dots, r_{m} \in K[x] \} \]
\end{defi}

\begin{prop}
	$ < f_{1}, \dots, f_{m} > $ is an ideal.
\end{prop} 

\begin{prop}
	$ 0 \in I $ (an ideal)
\end{prop}

\textbf{Problem : Ideal membership problem}. Given $ I=< f_{1}, \dots, f_{m} > $ and a polynomial $ h $. Determine $ h \in I $ or not ! 

\subsection{Single Variable}

Take $ n=1, x= x_{1}, K[x]=K[x_{1}] $. For $ f \in K[x] $ we define \textbf{degree of $ f $} as
\[ deg(f) = \begin{cases}
tdeg(f) , (f\neq0) \\ -\infty, (f=0)
\end{cases} \]
We define this such that properties below is satisfied.

\begin{prop}
	Let $ f,g \in K[x] $.
	\begin{enumerate}
		\item $ deg(f+g) \leq max \{ deg(f), deg(g) \} $
		\item $ deg)(fg) = deg(f)+deg(g) $
	\end{enumerate}
\end{prop}

\begin{ex}
	\begin{enumerate}
		\item $ f=2x^2 +1 , g =x+1 $
		\item  $ f=x+1, g=-x $
		\item  $ f=x+1, g =0 $
	\end{enumerate}
\end{ex}

\begin{theo}
	\textbf{Division Principle}. Let $ f,g \in K[x] $ and $ g \neq 0 $. Then there exist unique polynomials $ q,r $ such that
	\[ f= q \cdot g + r  \] and $ deg(r) < deg(g) $ where $ q $ is \textbf{quotient} and $ r $ is \textbf{remainder}.
\end{theo}

\begin{ex}
	$ f = x^3 +x -1 , g =2x^2 -1 $. Then $ f = x^3+x-1 = \dfrac{1}{2} x (2x^2-1) + \dfrac{3}{2}x -1 $ with $ deg(g)=2 , deg(r)=1 $
\end{ex}

\end{document}