\documentclass[a4paper,10pt]{article}
\usepackage[a4paper, hmargin={1.5cm,1.5cm}, vmargin={1.5cm,1.5cm}]{geometry}
\usepackage{amsmath}
\usepackage{amssymb}
\usepackage{amsthm}
\usepackage{amsfonts}
\usepackage{color}
\usepackage{graphicx}

\newtheorem{remark}{Remark}
\newtheorem{prop}{Proposition}
\newtheorem{theo}{Theorem}
\newtheorem{defi}{Definition}
\newtheorem{ex}{Example}
\newtheorem{note}{Notes}

\newcommand{\R}{\mathbb{R}}
\newcommand{\N}{\mathbb{N}}
\newcommand{\Z}{\mathbb{Z}}
\newcommand{\Q}{\mathbb{Q}}
\newcommand{\C}{\mathbb{C}}
\newcommand{\bx}{\mathbf{x}}
\newcommand{\by}{\mathbf{y}}

\begin{document}

\section{10-04-18}

\subsection{Algebraic axioms for real numbers}
Two binary operations, $ + $ addition and $ \cdot $ multiplication on $ \R $ are defined and have the following propoerties for all $ x,y,z \in \R $:
\begin{enumerate}
	\item $ x+(y+z)=(x+y)+z $. Associative law for addition.
	\item $ \exists 0 $ such that $ x+0=0+x=x $. Existence of additive identity.
	\item There exist an element $ -x\in\R $ such that $ x+ (-x) = (-x) + x = 0 $. Existence of additive inverse.
	\item $ x+y = y+x $. Commutative law for addition.
	\item $ x\cdot(y\cdot z)=(x\cdot y)\cdot z $. Associative law for multiplication.
	\item $ \exists 1 \neq 0 $ such that $ x \cdot 1 = 1 \cdot x = x $. Existence of multiplicative identity.
	\item If $ x \neq 0 $, then there exist an element $ x^{-1}\in\R $ such that $ x\cdot x^{-1} = x^{-1} \cdot x = 1 $. Existence of multiplicative inverse.
	\item $ x\cdot y = y\cdot x $. Commutative law for multiplication.
	\item $ x \cdot (y+z) = x \cdot y + x \cdot z $. Distributive law.
\end{enumerate}
In the language of algebra, axioms above state that $ \R $ with addition and multiplication is a \textbf{field}.

\subsection{The order axioms for real number}
A binary relation $ \leq $ on $ \R $ is defined and satisfies the following properties for all $ x,y,z \in \R $.
\begin{enumerate}
	\item $ x \leq x $. Reflexivity.
	\item If $ x \leq y, \ y \leq x $ then $ x=y $. Antisymmetry.
	\item If $ x\leq y, \ y \leq z $ then $ x \leq z $. Transitivity.
	\item Either $ x\leq y $ or $ y \leq x $. Totality.
	\item If $ x \leq y $, then $ x+y \leq y+z $
	\item If $ 0 \leq x $ and $ 0 \leq y $, then $ 0 \leq x \cdot y $.
\end{enumerate} 

\newpage
\section{17-04-18}

\subsection{Real Number}
$ \Q = \{ \dfrac{n}{m} | n,m \in \Z, m \neq 0 \} $. We have $ p,q \in \Q $, then
\[ p+q = \dfrac{n}{m} +\dfrac{k}{l} = \dfrac{kn + ml}{mk}; \ pq= \dfrac{nl}{mk}; \ p \geq q \Leftrightarrow p-q \geq 0 \]
For $ +, \times, \geq $ satisfy A1-A15.

\begin{remark}
	$ \Q $ is incomplete in the following sense. There is no $ r \in \Q$ such that $ r^2=2 $. Remember Phytagoras theorem, $ a^2+b^2=c^2 $. Pict : \vspace{2cm}
	$ \because $ if $ c \in \Q $, then $ c=\dfrac{n}{m} \ (n,m \in \Z, m \neq 0) $. We may assume that either $ m $ or $ n $ is odd.
	\[ c^2=2 \rightarrow \big(\dfrac{n}{m}\big)^2 =2 \rightarrow n^2=2m^2 \]
	\textbf{case 1 :} $ n $ is odd $ \Rightarrow $ odd = even (impossible)\\
	\textbf{case 2 :} $ n $  is even $ \Rightarrow m $ is odd (from assumtion) $ \Rightarrow n^2 $ can be devided by $ 4 $ but $ 2m^2 $ can not devided by $ 4 $ (contradiction)
\end{remark}

\textbf{Question :} How to fill the gap of $ \Q $ ? Answer : Idea of Weirstrass (supreme axioms)

\begin{defi}
	$ A \subset \R $.
	\begin{itemize}
		\item $ A $ is \underline{bounded from above} $ \Leftrightarrow \exists b\in\R $ such that $ a \leq b \ (\forall a\in A) $. such $ b $ is called \underline{upper bound of $ A $}.
		\item $ A $ is \underline{bounded from below} $ \Rightarrow \exists b\prime \in\R $ such that $ a \geq b\prime \ (\forall a \in A) $. Such $ b\prime $ is called \underline{lower bound of $ A $}
		\item $ \alpha = sup A $\\
		$ \Leftrightarrow $ the minimum of the set of upper bound\\
		$ \Leftrightarrow $ 1. $ \alpha $ is an upper bound of $ A $ ; 2. if $ b $ is an upper bound of $ A $, then $ \alpha \leq b $.
		\item $ \beta = inf A \Leftrightarrow $ the maximum of the set of lower bounds of $ A $.
	\end{itemize}
\end{defi}

\begin{remark}
	$ sup A (inf A) $ is uniquely determined if it exist. For example, $ sup \Q (inf \Q) $ does not exist. $ \because \Q $ is not bounded from above (below) 
\end{remark}

\begin{remark}
	\textbf{Completeness axioms}. Every nonempty subset of $ \R $ which is bounded from above (below) has a supremum (infimum) in $ \R $
\end{remark}

\subsection{Real sequence}
\begin{defi}
	For $ x\in \R, \ |x| = \begin{cases}
	x &, x \geq 0 \\ -x &, x \leq 0
	\end{cases} $
\end{defi}

\begin{remark}
	\begin{itemize}
		\item $ |x| \geq 0, \ |x|=0 \Leftrightarrow x=0 $
		\item $ |xy| = |x| |y| $
		\item $ |x+y| \leq |x| + |y| $ (triangle inequality)
	\end{itemize}
\end{remark}

An infinite sequence of $ \R \Leftrightarrow a: \N \rightarrow \R $ usually we write $ a_{n}=a(n) , \ n \in \N $ or $ \{ a_{n} \}_{n\in \N} $ or $ a_{1}, a_{2}, \dots $.\\
\textbf{Question :} Limiting behavior of $ a_{n} $ as $ n $ increases ?\\ 
Answer : $ a_{n} \rightarrow l, \ n \rightarrow \infty \Leftrightarrow $ as $ n $ become larger and larger, the value $ a_{n} $ become arbitrarily close to $ l $.

\begin{defi}
	\textbf{$ \epsilon-N $ definition of the limit}. $ \{ a_{n} \} $ converges to $ l \in \R \Leftrightarrow \forall \epsilon>0, \exists N \in \N $ such that $ |a_{n}-l| < \epsilon, \forall n \geq N $. We write $ \lim\limits_{n \rightarrow \infty} a_{n} = l $.
\end{defi}

\begin{defi}
	\begin{itemize}
		\item $ a_{n} \rightarrow +\infty \Leftrightarrow \forall M>0. \ \exists N \in \N $ such that $ a_{n}>M \ (\forall n\geq N)$
		\item $ a_{n} \rightarrow -\infty \Leftrightarrow \forall M>0. \ \exists N \in \N $ such that $ a_{n}<-M \ (\forall n\geq N)$
	\end{itemize}
\end{defi}

\begin{remark}
	A convergent sequence has a unique limit.\\
	\begin{eqnarray} \nonumber
	\because & \epsilon = \dfrac{1}{2}|l-l\prime|>0 \\ \nonumber
	& \exists N\in\N \text{ such that } |a_{n}-l| < \epsilon, \ (\forall n \geq N) \\ \nonumber
	& \exists N\prime\in\N \text{ such that } |a_{n}-l\prime| < \epsilon, \ (\forall n \geq N\prime)
	\end{eqnarray}
	Set $ \tilde{N} = max \{ N,N\prime \} \in \N $. For $ n\geq \tilde{N} \Rightarrow |a_{n}-l|<\epsilon, \ |a_{n}-l\prime|<\epsilon $ is impossible.
\end{remark}

\newpage 
\textbf{REPORT 1}

\textbf{Afifah Maya Iknaningrum (1715011053)}

\begin{enumerate}
	\item \underline{Problem :} Let $ \{a_{n}\}, \{b_{n}\}, \{c_{n}\} $ be a real sequence. Suppose that for every $ n \in \N $, we have
	\[ b_{n} \leq a_{n} \leq c_{n} \]
	and also suppose that
	\[ \lim\limits_{n \rightarrow \infty} b_{n} = l = \lim\limits_{n \rightarrow \infty} c_{n} \]
	for some $ l \in \R $. Then \[ \lim\limits_{n \rightarrow \infty} a_{n} = l. \]\\
	\underline{Answer :} By definition of limit, $ \forall \epsilon>0 , \ \exists N_{1}, N_{2} \in \N $ such that for $ l \in \R $
	\[ |b_{n}-l| < \epsilon, \ \forall n \geq N_{1}, \]
	\[ |c_{n}-l| < \epsilon, \ \forall n \geq N_{2}. \]
	Then we can obtain
	\[ |b_{n}-l| < \epsilon \Leftrightarrow -\epsilon < b_{n}-l < \epsilon \Leftrightarrow l-\epsilon < b_{n} < l + \epsilon,\]
	\[ |c_{n}-l| < \epsilon \Leftrightarrow -\epsilon < c_{n}-l < \epsilon \Leftrightarrow l-\epsilon < c_{n} < l + \epsilon.\]
	Take $ N = max \{N_{1},N_{2}\} $, then $ \forall n > N $
	\begin{eqnarray}\nonumber
	 && b_{n} \leq a_{n} \leq c_{n} \\ \nonumber
	 & \Leftrightarrow & l-\epsilon < b_{n} \leq a_{n} \leq c_{n} < l + \epsilon \\ \nonumber
	 & \Leftrightarrow & l-\epsilon < a_{n} < l + \epsilon\\ \nonumber
	 & \Leftrightarrow & -\epsilon < a_{n}-l < \epsilon \\ \nonumber
	 & \Leftrightarrow & |a_{n}-l| < \epsilon.
	\end{eqnarray}
	It is proved that $ \forall \epsilon>0 , \ \exists N \in \N $ such that for $ l \in \R $
	\[ |a_{n}-l| < \epsilon, \ \forall n \geq N \]
	or we can write
	\[ \lim\limits_{n \rightarrow \infty} a_{n} = l. \] \qed
	\newpage \item 
	\begin{enumerate}
		\item \underline{Problem :} If a sequence of real numbers converges, then it is bounded.\\
		\underline{Answer :} Let $ \{x_{n}\} $ be a sequence in real number. Suppose $ \{x_{n}\} $ is converge to $ a \in \R $ as $ n \rightarrow \infty $. Then $ \forall \epsilon > 0, \ \exists N\in \N $ such that $ \forall n > N $, \[|x_{n}-a|<\epsilon.\]
		From triangle inequality we obtain
		\begin{eqnarray} \nonumber
		|x_{n}-a|&<&\epsilon \\ \nonumber
		|x_{n}|-|a| &<& \epsilon \\ \nonumber
		|x_{n}| &<& \epsilon + |a|.
		\end{eqnarray}
		Takes $ M = max\{ \epsilon + |a|, x_{1}, x_{2}, \dots, x_{N} \} $, we obtain
		\[ |x_{n}| \leq M. \]
		It shows that $ \forall \epsilon>0, \exists M>0 $ such that $ |x_{n}| \leq M, \ \forall n $ or it is proved that $ \{x_{n}\} $ is bounded. \qed
		\item \underline{Problem :} If a sequence of real numbers converge, then it is a Cauchy sequence.\\
		\underline{Answer :} Let $ \{x_{n}\}, \{x_{m}\} $ be a sequence in real number. Suppose $ \{x_{n}\}, \{x_{m}\} $ is converge to $ a \in \R $ as $ n \rightarrow \infty $. Then $ \forall \epsilon > 0, \ \exists N_{1},N_{2}\in \N $ such that $ \forall n > N_{1} $, \[|x_{n}-a|<\dfrac{\epsilon}{2}\]
		and  $ \forall m > N_{2} $, \[|x_{m}-a|<\dfrac{\epsilon}{2}.\]
		Takes $ N=max\{N_{1},N_{2}\} $ then $ \forall n, m > N $
		\begin{eqnarray}\nonumber
		|x_{n}-x_{m}| & \leq & |x_{n}-a+a-x_{m}| \\ \nonumber
		& \leq & |x_{n}-a|+|x_{m}-a| \\ \nonumber
		& < & \dfrac{\epsilon}{2}+\dfrac{\epsilon}{2} \\ \nonumber
		& < & \epsilon.		
		\end{eqnarray}
		Then, it is proved that $ \forall \epsilon>0, \ \exists N \in \N $ such that for $ n,m > N $ \[ |x_{n}-x_{m}|<\epsilon \] or it is Cauchy sequence. \qed
	\end{enumerate}
\end{enumerate}

\newpage
\section{07-05-2018}

\subsection{Landau Symbol}
Symbol for representing the behavior of functions. $ O $ : big o and $ o $ : small o.\\
Let $ f,g $ be function around $ x=a \in \R $ (or $ x>M $ for some $ M \in \R $)
\begin{itemize}
	\item $ f(x) = O(g(x)) $ as $ x \rightarrow a $ if $ \exists \delta>0, \exists A>0 $ such that $ |f(x)| \leq A \ g(x) $ for $ 0 < |x-a| < \delta $.\\ Means : eventually the graph $ f(x) $ is below of $ y=A \ g(x) $.
	\item $ f(x) = O(g(x)) $ as $ x \rightarrow a $ if $ \exists m > M, \exists A>0 $ such that $ |f(x)| \leq A \ g(x) $ for $ x>m $. \\ Means : $ |f(x) $ is eventually \textit{dominated} by linear function as $ x \rightarrow \infty $
\end{itemize}

\underline{Example :} \\$ f(x) = O(x^2), \ (x\rightarrow \infty) $ then $ f(x) $ is eventually dominated by a quadratic function as $ x \rightarrow \infty $.\\
$ f(x) = O(1) $ as $ x \rightarrow \infty $ then $ f(x) $ is a bounded function around $ \infty $

\underline{Explanation behavior :} $ f(x) $ is a polynomial time behaviour as $ x \rightarrow \infty $. $ f(x) = O(e^{ax}), \ f(x) = o(x^n) $ for some $ n \in \N $.
\begin{itemize}
	\item $ f(x) = o(g(x)), \ (x \rightarrow \infty) $ if $ \forall\epsilon>0, \exists \delta >0 $ such that $ |f(x)| \leq \epsilon g(x), 0 < |x-a| < \delta $ \\ (or equivalently, if $ g(x)\neq 0, \ \lim\limits_{x\rightarrow a} \dfrac{f(x)}{g(x)} =0 $)
	\item $ f(x) = o(g(x)), \ (x \rightarrow \infty) $ if $ \forall\epsilon>0, \exists m >0 $ such that $ x>m \Rightarrow |f(x)| \leq \epsilon g(x) $ \\ (or equivalently, if $ g(x)\neq 0, \ \lim\limits_{x\rightarrow \infty} \dfrac{f(x)}{g(x)} =0 $)
	\item $ f(x) = o(x) $ as $ x \rightarrow \infty \Leftrightarrow \lim\limits_{x\rightarrow \infty} \dfrac{f(x)}{x} =0$
	\item $ f(x) = o(x) $ as $ x \rightarrow 0 \Leftrightarrow \lim\limits_{x\rightarrow 0} \dfrac{f(x)}{x} =0$
	\item $ f(x) = o(1) $ as $ x \rightarrow \infty \Leftrightarrow \lim\limits_{x \rightarrow \infty} \dfrac{f(x)}{1} =0 \Leftrightarrow \lim\limits_{x \rightarrow \infty} f(x) =0 $ 
	\item $ f(x) = o(1) $ as $ x \rightarrow a \ (a \in \R)  \Leftrightarrow \lim\limits_{x \rightarrow a} f(x) =0 $ 	
\end{itemize}

\begin{remark}
	\begin{enumerate}
		\item $ f(x) $ is continuous at $ x=a \Leftrightarrow \lim\limits_{x \rightarrow \infty} f(x) =f(a) $ iff $ \Leftrightarrow \lim\limits_{x \rightarrow a} (f(x)-f(a)) =0 $\\
		by previous, $ \Leftrightarrow f(x) - f(a) = o(1) $ as $ x \rightarrow a \Leftrightarrow f(x) = f(a) +o(1) $ as $ x \rightarrow a $ 
		\item $ f(x) $ is differentiable as $ x=a \Leftrightarrow \lim\limits_{h\rightarrow 0}\dfrac{f(a+h)-f(a)}{h} = f'(a) $ iff $ \Leftrightarrow \dfrac{f(a+h)-f(a)}{h} = f'(a) + o(1) $ as $ h \rightarrow 0 $\\
		$ \Leftrightarrow f(a+h)=f(a)+f'(a) \ h + o(h) $ as $ h\rightarrow 0 $\\
		note : $ o(h) \Leftrightarrow \dfrac{o(h)}{h}=0 \Rightarrow o(1) \ h \Rightarrow \dfrac{o(1) \ h}{h} = o(1) \rightarrow 0 $ as $ h\rightarrow 0 $
	\end{enumerate}
\end{remark}

$ D : $ domain in $ \R^2 $ and $ f : $ function on $ D $
\[ f_{x}(a,b) = \dfrac{\partial f}{\partial x} (a,b) = \lim\limits_{h\rightarrow 0} \dfrac{f(a+h,b)-f(a,b)}{h} \]
\[ f_{y}(a,b) = \dfrac{\partial f}{\partial y} (a,b) = \lim\limits_{k\rightarrow 0} \dfrac{f(a,b+k)-f(a,b)}{k} \]
\[ f(a+h,b) = f(a,b) +f_{x}(a,b)h + o(h) \text{ as } h\rightarrow 0 \]
\[ f(a,b+k) = f(a,b) +f_{y}(a,b)k + o(k) \text{ as } k\rightarrow 0 \]

\newpage
\textbf{REPORT 2 \\ Afifah Maya Iknaningrum (1715011053)}

\underline{Problem 1 :} Let $ C([a,b]) $ be the set of all continuous functions $ f : [a,b] \rightarrow \R $ and define
\[ d_{2}(f,g) := \Big[ \int_{a}^{b} \big( f(x)-g(x) \big)^{2} dx  \Big]^{1/2} \]
for $ f,g \in C([a,b]) $. Show that $ (C([a,b]),d_{2}) $ is a metric space.\\ \\
\underline{Answer :}\\
To proof that $ (C([a,b]),d_{2}) $ is a metric space, we need to proof :
\begin{enumerate}
	\item $ d_{2}(f,g) \geq 0 $ and $ d_{2}(f,g)=0 \Leftrightarrow f=g $. \\ \\
	\underline{Proof : } \\
	By the definitions of $ d_{2}(f,g) $, it is obvious that the value of integral is always positive. So it is proved that $ d_{2}(f,g) \geq 0 $.\\
	Then, \\
	$ (\Rightarrow) $ We have 
	\[d_{2}(f,g)= \Big[ \int_{a}^{b} \big( f(x)-g(x) \big)^{2} dx  \Big]^{1/2} =0\]
	The only possible answer will be
	\[ f(x)-g(x) =0 \text{ or } f(x)=g(x) \]
	$ (\Leftarrow) $ We have $ f(x)=g(x) $, using the definition of $ d_{2}(f,g) $
	\begin{eqnarray}\nonumber
	d_{2}(f,g) &=& \Big[ \int_{a}^{b} \big( f(x)-g(x) \big)^{2} dx  \Big]^{1/2} \\ \nonumber
	&=& \Big[ \int_{a}^{b} 0 \ dx  \Big]^{1/2} \\ \nonumber
	&=& 0
	\end{eqnarray} 
	\item $ d_{2}(f,g) = d_{2}(g,f) $. \\ \\
	\underline{Proof : } \\
	\begin{eqnarray}\nonumber
	d_{2}(f,g) &=& \Big[ \int_{a}^{b} \big( f(x)-g(x) \big)^{2} dx  \Big]^{1/2} \\ \nonumber
	&=& \Big[ \int_{a}^{b} \big( g(x)-f(x) \big)^{2} dx  \Big]^{1/2} \\ \nonumber
	&=& d_{2}(g,f)
	\end{eqnarray}
	\item $ d_{2}(f,g) \leq d_{2}(f,h) + d_{2}(h,g) $. \\ \\
	\underline{Proof :} \\
	Using fact that
	\[\int\big(a+b\big)^2 = \int a^2 + \int b^2 + 2 \int ab \]
	and via Schwartz inequality
	\[ \int ab \leq \sqrt{\int a^2} \sqrt{\int b^2} \]
	then
	\[ \int (a+b)^2 \leq \int a^2 + \int b^2 + 2 \sqrt{\int a^2} \sqrt{\int b^2} = \Big(\sqrt{\int a^2} + \sqrt{\int b^2}\Big)^2 \]
	Using these fact with $ a=f-h $ and $ b=h-g $,
	\begin{eqnarray}\nonumber
	d_{2}(f,g) &=& \Big[ \int_{a}^{b} \big( f(x)-g(x) \big)^{2} dx  \Big]^{1/2} \\ \nonumber
	&=& \Big[ \int_{a}^{b} \big( f(x)-h(x)+h(x)-g(x) \big)^{2} dx  \Big]^{1/2} \\ \nonumber
	&\leq& \Big(\Big[ \int_{a}^{b} \big( f(x)-h(x) \big)^{2} dx  \Big]^{1/2} + \Big[ \int_{a}^{b} \big( h(x)-g(x) \big)^{2} dx  \Big]^{1/2}\Big)^{2(1/2)} \\ \nonumber
	&\leq&  \Big[ \int_{a}^{b} \big( f(x)-h(x) \big)^{2} dx  \Big]^{1/2} + \Big[ \int_{a}^{b} \big( h(x)-g(x) \big)^{2} dx \Big]^{1/2} \\ \nonumber
	&\leq& d_{2}(f,h) + d_{2}(h,g)
	\end{eqnarray}
\end{enumerate} \qed

\underline{Problem 2 :} Let $ (X,d) $ be a metric space. Prove that the function
\[ \tilde{d}(x,y) := \dfrac{d(x,y)}{1+d(x,y)} \ , \ \ (x,y\in X)   \]
is also a metric on $ X $. \\ \\
\underline{Answer :} \\
It is known that $ (X,d) $ is metric space. Then for $ x,y,z \in X $ we have the following
\begin{enumerate}
	\item $ d(x,y) \geq 0 $ and $ d(x,y)=0 \Leftrightarrow x=y $
	\item $ d(x,y) =d(y,x) $
	\item $ d(x,z) \leq d(x,y) + d(y,z) $.
\end{enumerate}
We want to proof that $ \tilde{d}(x,y) := \dfrac{d(x,y)}{1+d(x,y)} $ is also a metric space. We need to proof :
\begin{enumerate}
	\item $ \tilde{d}(x,y) \geq 0 $ and $ \tilde{d}(x,y)=0 \Leftrightarrow x=y $. \\ \\
	\underline{Proof : } \\
	Using the fact that $ d(x,y) \geq 0 $, it is obvious that 
	\[ \tilde{d}(x,y) = \dfrac{d(x,y)}{1+d(x,y)} \geq 0. \]
	$ (\Rightarrow) $  \[\tilde{d} (x,y) = \dfrac{d(x,y)}{1+d(x,y)} =0\] only possible if $ \tilde{d}(x,y)=0 $. Using properties of $ (X,d) $,  \[d(x,y)=0 \Leftrightarrow x=y \], then it proved that $ \tilde{d}(x,y)=0 \Leftrightarrow x=y $. \\
	$ (\Leftarrow) $ For $ x=y $, using the fact $ d(x,y)=0 \Leftrightarrow x=y $, we have
	\[  \tilde{d}(x,y) = \dfrac{d(x,y)}{1+d(x,y)} =0 \]
	\item $ \tilde{d}(x,y) = \tilde{d}(y,x) $. \\ \\
	\underline{Proof : } \\
	Because $ (X,d) $ is metric space, then $ d(x,y) =d(y,x) $ is hold. Such that
	\begin{eqnarray}\nonumber
	\tilde{d}(x,y) &=& \dfrac{d(x,y)}{1+d(x,y)} \\ \nonumber
	&=& \dfrac{d(y,x)}{1+d(y,x)}\\ \nonumber
	&=& \tilde{d}(y,x)
	\end{eqnarray}
	\item $ \tilde{d}(x,z) \leq \tilde{d}(x,y) + \tilde{d}(y,z) $. \\ \\
	\underline{Proof :}\\
	Using triangle inequality of $ (X,d) $,
	\begin{eqnarray}\nonumber
	\tilde{d}(x,z) &=& \dfrac{d(x,z)}{1+d(x,z)} \\ \nonumber
	&\leq&  \dfrac{d(x,y)+d(y,z)}{1+d(x,y)+d(y,z)} \\ \nonumber
	&\leq& \dfrac{d(x,y)}{1+d(x,y)+d(y,z)} +\dfrac{d(y,z)}{1+d(x,y)+d(y,z)} \\ \nonumber
	&\leq& \dfrac{d(x,y)}{1+d(x,y)} +\dfrac{d(y,z)}{1+d(y,z)} \\ \nonumber
	&\leq& \tilde{d}(x,y) + \tilde{d}(y,z)
	\end{eqnarray}
\end{enumerate}\qed

\newpage
\textbf{REPORT 3 \\ Afifah Maya Iknaningrum (1715011053)}\\ \\

\underline{Problem 1 :} Prove that a subset of a metric space is open if and only if it is a union of open balls. \\ \\
\underline{Answer :} \\ \\
$ (\Rightarrow) $Suppose $ G $ in $ (X,d) $ is open. If $ G $ is empty, there no open balls contained in it. Thus union of an empty class, which is empty and therefore equal to $ G $. If $ G $ is nonempty, then $ G $ is open such that $ \forall x \in G , \exists r>0, B_{r}(x) \subset G $ then $ G = \bigcup \ B_{r}(x) $.\\ \\
$ (\Leftarrow) $ In metric space, it is known that every open ball is open set. And, union of open set is open.\\
Let $ G = \bigcup\limits_{\alpha \in \Lambda} B_{r}(\alpha) $ for $ \alpha \in G, \exists r>0 $. If $ G $ is empty, then it is open. So we assume $ G $ is nonempty. Consider $ y \in G $, then $ y \in B_{r}(\alpha) $ for some $ \alpha \in \Lambda $. Since $ B_{r}(\alpha) $ is open, $ \exists r > 0 $ such that $ B_{r}(y) \subseteq B_{r}(\alpha) \subseteq G $. \\
Thus $ \forall y \in G , \exists r>0 $ such that $ B_{r}(y) \subseteq G $. Consequently, $ G $ is open. \\ \\ 

\underline{Problem 2 :} Let $ C([0,1]) $ be the set of all continuous function $ f : [0,1] \rightarrow \R $ and define 
\[ d_{1}(f,g) := \int_{0}^{1} |f(x)-g(x)| \ dx \]
for $ f,g \in C([0,1]) $. Show that $ (C([0,1]),d_{1}) $ is not complete.\\
Hint : Consider the sequence $ \{f_{n}\}_{n\geq 3} $ defined by
\[ f_{n}(x) = \begin{cases}
0 &, 0 \leq x < \frac{1}{2} - \frac{1}{n}, \\
n(x+\frac{1}{n}-\frac{1}{2}) &,  \frac{1}{2} - \frac{1}{n} \leq x < \frac{1}{2} \\
1 &, \frac{1}{2}\leq x \leq 1
\end{cases} \]\\ \\
\underline{Answer :} \\ \\
Considering the sequence $ \{f_{n}\}_{n\geq 3} $ above, then
\[ \|f_{n}-f_{m}\| = \Big(\int_{1/2-1/n}^{1/2} \| f_{n}(x)-f_{m}(x)\| \ dx\Big) \leq \big(\dfrac{-1}{n}\big) \rightarrow 0  \]
so $ f_{n} $ is Cauchy. Suppose $ f_{n} $ has limit $ f \in C([0,1]) $. Then
\[ \int_{1/2}^{1} |f(x)-f_{n}(x)| \ dx \leq \|f-f_{n}\| \rightarrow 0 \]
so $ f(x) =1 $ on $ [1/2,1] $. Similarly we see $ f(x) = 0 $ on $ [0,1/2] $ which is contradiction.


\newpage
\textbf{REPORT 4 \\ Afifah Maya Iknaningrum (1715011053)}\\ \\

\begin{enumerate}
	\item Let $ (X,d) $ be a complete metrix space and let $ f :X \rightarrow X $ be a map. Suppose that the iterated map
	\[ f^{k} = f \circ\dots\circ f \ \ \text{ (k times)} \]
	is a contraction for some $ k\geq2 $. ($ f $ is not necessarily a contraction.) Prove that $ f $ has a  unique fixed point $ x\in X $.\\ \\
	\textbf{Answer :}\\
	Let $ x $ be a fixed point of $ f^{k} $ such that
	\[ f^{k}(x) = x \ \text{ and } f^{k+1}(x) =  f(x) \]
	so $ f^{k}(f(x)) =f(x) $.
	Hence $ f(x) $ is also a fixed point. By uniqueness, $ f(x)=x $.
	\item Complete the proof of Picard-Lindelof Theorem by showing that $ A : M\rightarrow M $ is a contraction if $ h $ is small.\\ \\
	\textbf{Answer :}\\
	By induction, we can show that $ | A(x_{n})-A(x_{n-1}) | \leq \dfrac{CA^{n-1}}{n!} |t-t_{0}|^{n} $. Because
	\[ A(x_{n}) - A(x_{n-1}) = \int (f(t,x_{n-1})-f(t,x_{n-2}) \]
	Next, we can see that $ A(x_{n}) $ is convergent uniformly for $ t_{0}-h \leq t \leq t_{0}+h $. Putting very small $ h $, we can see that $ A(x)=x $
\end{enumerate}

\end{document}